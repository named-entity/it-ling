% ----------------------------------------------------------------
% Article Class (This is a LaTeX2e document)  ********************
% ----------------------------------------------------------------
\documentclass[12pt]{article}
\usepackage[english, russian]{babel}
\usepackage{amsmath,amsthm}
\usepackage{amsfonts}
\usepackage{indentfirst}
\usepackage{enumitem}
\usepackage[colorlinks,urlcolor=blue]{hyperref}
\hypersetup{
    colorlinks,
    citecolor=black,
    filecolor=black,
    linkcolor=black,
    urlcolor=black
}
\usepackage[utf8]{inputenc}

% THEOREMS -------------------------------------------------------
\newtheorem{thm}{Theorem}[section]
\newtheorem{cor}[thm]{Corollary}
\newtheorem{lem}[thm]{Lemma}
\newtheorem{prop}[thm]{Proposition}
\theoremstyle{definition}
\newtheorem*{defn}{Определение}
\theoremstyle{remark}
\newtheorem{rem}[thm]{Remark}
\numberwithin{equation}{section}
% ----------------------------------------------------------------
\begin{document}

\title{Информационные технологии в лингвистике}%
\author{Билеты к экзамену}%
\date{}
% ----------------------------------------------------------------
\maketitle
% ----------------------------------------------------------------
\tableofcontents

\section{Лингвистические информационные технологии; лингвистические ресурсы в Интернете}
\subsection{Информационные технологии в лингвистике}
В~70-е годы XX века началось бурное развитие компьютерных технологий.
Компьютеры позволили обрабатывать огромные массивы информации, прежде
неподвластные обработке. Это дало скачок в~развитии некоторых областей
прикладной лингвистики, ранее находившихся в~зачаточном состоянии, и
позволило по-новому взглянуть на~ряд фундаментальных задач.
\begin{defn}
    {\sl Информационные технологии в лингвистике}~--- это совокупность
    законов, методов и средств получения, хранения, передачи,
    распространения, преобразования информации о~языке и законов
    его функционирования с~помощью компьютеров.
\end{defn}

Информационные технологии находят свое применение прежде всего
в~прикладных задачах. К~их числу можно отнести:
\begin{itemize}
    \item создание систем искусственного интеллекта;
    \item создание систем автоматического перевода;
    \item создание систем автоматического аннотирования и реферирования
    текстов;
    \item создание систем порождения текстов;
    \item создание систем обучения языку;
    \item создание систем понимания устной речи;
    \item создание систем генерации речи;
    \item создание автоматизированных информационно-поисковых систем;
    \item создание систем атрибуциии дешифровки анонимных и
    псевдоанонимных текстов;
    \item разработка различных баз данных (словарей, карточек, каталогов
    реестров, и~т.п.) для гуманитарных наук;
    \item разработка различного типа автоматических словарей;
    \item разработка систем передачи информации в~сети Интернет и~т.п.
\end{itemize}

Самое слово {\sl информация} означает некоторые сведения о~внешнем и
внутреннем мире, которые мы используем для~регулирования своего
поведения. Она определяется такими параметрами: ценность, достоверность,
полнота, актуальность, логичность, компактность.

С~современной точки зрения {\sl информатика}~--- наука о~законах и
методах получения, хранения, передачи, распространения и использования
информации в~естественных и искусственных системах с~применением
компьютера.

\subsection{Лингвистические ресурсы в интернете}
Вот некоторые лингвистические ресурсы в интернете:
\begin{enumerate}[label=\*]
    \item \href{http://linguistlist.org/}{The Linguist List}~--- большой
    каталог программного обеспечения для~различных областей компьютерной
    обработки текстов и лингвистики.
    \item \url{http://www.sil.org/computing/}~--- обширный каталог
    программ по~вычислительной лингвистике, разработанных в~рамках
    проекта SIL.
    \item \url{http://www.sil.org/linguistics/computing.html}~--- очень
    большая коллекция ссылок на~программы чрезвычайно широкой лингвистической
    направленности в~сети Internet.
    \item \url{http://www.content-analysis.de/index.html}~--- электронный
    каталог ресурсов, связанных с~анализом текстов (на~англ.языке).
    \item \href{http://www.gramota.ru/index.html}{Справочно-информационный
    портал <<Русский язык>>}~--- замечательный ресурс для~истинных
    любителей русской словесности, содержащий массу полезной информации.
    Также включает on-line словари русского языка.
    \item \href{http://www.textanalysis.info/}{Text Analysis Info}~---
    бесплатный информационный портал, посвященный анализу контента
    межчеловеческого общения. Также представлены различные программы,
    предназначенные для обработки таких источников как аудио-, видео- или
    речевых данных.
    \item \href{http://www.lti.cs.cmu.edu/research/projects}{LTI Projects}~---
    каталог проектов, посвященных созданию систем машинного перевода,
    обработки речи, информационного поиска, извлечения знаний и других.
    \item \url{http://ruscorpora.ru/}~--- национальный корпус русского языка.
    \item \url{http://opencorpora.org/}~--- открытый корпус русского языка.
\end{enumerate}

\section{Типология программных средств для лингвистических информационных технологий}
Существует немало программных средств, предназначенных для~обеспечения
анализа, обработки, хранения и поиска текстов, рисунков и аудиоданных
на~естественном языке.
\paragraph{Программы анализа и лингвистической обработки текстов.}
Наиболее важные парсеры этого типа производят морфологическую разметку
текста, по~возможности снимая неоднозначность на~разных уровнях, или
строят графы зависимостей, выделяют имена собственные и решают более
узконаправленные задачи.

\paragraph{Программы преобразования текстов.}
Решают задачи поиска и замены в~тексте определенных элементов,
смены кодировки, обработку разметки и т.п.
\paragraph{Проверка орфографии.}
Ряд парсеров проверяет пунктуацию, орфографию, грамматику и оценивает
стилистику текста.
\paragraph{Автоматические генераторы текстов}
Еще некоторое время назад многие подобные программы имели шуточный
характер: они генерировали стихи или <<философские>> изречения,
но сейчас и это направление бурно развивается. Крупные компании
используют автоматические генераторы канцелярских документов, отчетов
и даже автореферирование и рецензирование.
\paragraph{Машинный перевод.}
Программы и сервисы, предназначенные для~автоматического перевода
текстов с~одного естественного языка на~другой.
\paragraph{Словари и тезаурусы}
Электронные словари, теперь бывают и статистические.
\paragraph{Поисковые машины и системы полнотекстового поиска.}
Поиск по~ключевой информации, кластеризация результатов.
\paragraph{Синтез и распознавание речи.}
Направление исследований: распознавание $\rightarrow$ перевод
$\rightarrow$ синтез.
\paragraph{Распознавание символов}

\section{Автоматический анализ текста; уровни анализа; взаимодействие
между уровнями}
Отражение уровневой организации языка в~архитектуре систем
компьютерного анализа естественно-языкового текста:
\begin{itemize}
    \item фонетический уровень;
    \item графематический уровень;
    \item морфологический уровень;
    \item синтаксический уровень;
    \item семантический уровень.
\end{itemize}

\paragraph{Фонетика}~--- раздел языкознания, изучающий речевые
звуки и звуковое строение языка (слоги, звукосочетания, закономерности
соединения звуков в речевую цепочку). Различает звуки гласные
и согласные, звонкие и глухие и т.д. Изучает:
\begin{itemize}
    \item звук речи с~точки зрения его создания, какие органы речи
    участвуют в~его произношении;
    \item звук как колебание воздуха и фиксирует его физические
    характеристики: частоту (высоту), силу (амплитуду), длительность;
    \item функции звуков в~языке, оперирует фонемами, т.е.
    минимальными языковыми единицами, обладающими смыслоразличительной
    функцией.
\end{itemize}
\paragraph{Графематический анализ.} Этап графематического
анализа предназначен для~выделения элементов структуры текста:
параграфов, абзацев, предложений, отдельных слов и~т.д.
В~задачу графематического анализа входят:
\begin{itemize}
    \item Выделение абзацев, заголовков, примечаний;
    \item Выделение предложений из~входного текста;
    \item Разделение входного текста на~слова, цифровые комплексы,
    формулы и~т.д. {\sl Токенизация};
    \item Сборка слов, написанных в~разрядку;
    \item Выделение устойчивых оборотов, не~имеющих словоизменительных
    вариантов;
    \item Выделение ФИО (фамилия, имя, отчество), когда имя и отчество
    написаны инициалами;
    \item Выделение иностранных лексем, записанных латиницей;
    \item Выделение электронных адресов и имен файлов;
\end{itemize}

\paragraph{Морфологический анализ.} Единица~--- слово. На~этом этапе
обрабатываются отдельные слова, в~них выделяются основы и флексии
(изменяемые части слов)~--- приставки, суффиксы, окончания.
В~дальнейшем флексии используются для~установления грамматических
отношений между словами в~рамках одного предложения.

Слова как единицы грамматические и лексические группируются
в~части речи, т.е. в грамматические классы слов, объединяющиеся
на~основании обобщенных значений. Обобщенное значение, характеризующее
все слова той или иной части речи, есть абстрактное представление
того общего, что присутствует в~лексических и морфологических значениях
конкретных слов данного класса. Наиболее обобщенными значениями
для~частей речи являются значения предмета (субстанции) и признака~---
процессуального (представляемого как действие или состояние) и
непроцессуального (представляемого как качество или свойство).

Морфологический анализ обеспечивает определение нормальной
формы, от~которой была образована данная словоформа, и набора
параметров, приписанных данной словоформе. Нормальная форма
(именительный падеж для~существительных, инфинитив для~глаголов
и~т.д.) называется {\sl леммой}, а сам процесс определения лемм~---
{\sl лемматизацией}.

\paragraph{Синтаксический анализ.} Единица~--- предложение.
В~результате синтаксического анализа линейная последовательность
токенов (слов) преобразуется в~набор синтаксических отношений.
Грамматично построенные предложения являются связными, т.е.
лишенными разрывов в~цепочке синтаксических отношений. Отношения
являются бинарными. Синтаксическое отношение неравноправно:
определяемое слово <<главнее>> своего определения. Важная
особенность синтаксических зависимостей заключается в~том,
что они далеко не~всегда связывают слова, находящиеся рядом
в~цепочке.

Выполнение задачи осложняется огромным количеством альтернативных
вариантов, возникающих в~ходе разбора, связанных как
с~многозначностью входных данных (одна и та же словоформа может
быть получена от~различных нормальных форм), так и неоднозначностью
самих правил разбора.

\paragraph{Семантический анализ.} Занимается решением задач,
связанных с~возможностью определения значения слова в~зависимости
от~контекста и конкретной ситуации, понимания смысла фразы.
Элемент значения языкового знака называется {\sl семой}.

Во многих случаях смысловой элемент состоит из нескольких слов.
Последовательность из двух или более слов, частотность совместного
появления которых в тексте выше, чем ожидаемая вероятность их
совместного появления, называется {\sl коллокацией}. В отличие
от свободного словосочетания, коллокация определяет, какие слова
могут быть использованы вместе, например, с какими предлогами
управляет тот, или иной глагол, или какие глаголы и существительные
обычно используются вместе.

С помощью компьютерных технологий коллокации могут автоматически
извлекаться из текстов. Для этого используются различные меры
ассоциативной связи, которые оценивают, является ли взаимное появление
лексических единиц случайным, или оно статистически значимо. Однако
часто статистически значимое совместное появление двух слов не образует
коллокации.

\section{Фонетический уровень анализа языка и речи}

Фонетический уровень описывает звуковую сторону языка. {\sl Фонетика}~---
наука о~<<звуковом материале>> языка, об~использовании этого
материала в~значащих единицах языка и речи, об~исторических изменениях
в~этом материале и в~приемах его использования. Звуки и другие звуковые
единицы (например, слоги), а также такие звуковые явления, как ударение
и интонация, изучаются фонетикой с~нескольких разных точек зрения:
\begin{itemize}
    \item {\sl акустическая фонетика}~--- изучает звуки как физические
    явления;
    \item с~точки зрения работы, производимой человеком при~их произнесении
    и слуховом восприятии, т.е. в~биологическом аспекте~--- {\sl артикуляторная
    фонетика};
    \item и самое главное~--- с~точки зрения их использования в~языке, т.е.
    отношений между собой в~речевой цепи~--- {\sl фонология}.
\end{itemize}

Фонетика имеет большое практическое значение. Без~нее были~бы
невозможны правильная методика обучения письму и чтению, постановка
произношения при~изучении неродного языка, создание рациональной
системы письма для~бесписьменных языков и усовершенствование
существующих систем письма, успешное лечение дефектов речи и~т.д.
При~изложении различных вопросов фонетики приходится во~многих
случаях пользоваться специальными видами письма~--- той или иной
научной транскрипцией. Это связано с~тем, что в~обычном письме
между буквой и звуком часто нет однозначного соответствия.
Например, в~русском письме один и тот~же звук нередко записывается
разными буквами (скажем, буквой {\bf в} в слове {\bf плечевой} и буквой {\bf г}
в слове {\bf чего}), и наоборот, одна и та~же буква читается как разные
звуки (скажем, буква {\bf г} в~словах {\bf игра}, {\bf чего},
{\bf легкий}, {\bf снег}).

\subsection{Методика записи устной речи}
Имеется два направления исследования речи:
\begin{itemize}
    \item Исследовать записанную дикторами речь, то есть произношение
    текста отдельными информантами,
    \item Исследовать спонтанную речь, записывая живые разговоры.
\end{itemize}
В этом пункте мы рассмотрим первый тип.

\subsection{Этапы исследования устной речи}
Создание речевых баз данных (или, иначе, речевых корпусов) представляет
собой определенный технологический процесс. В~нем можно выделить следующие
основные этапы:
\begin{itemize}
    \item подготовка фонетического обеспечения для~формирования речевого
    корпуса, транскрипция;
    \item подготовка текстового материала;
    \item разработка программного обеспечения для~формирования речевого
    корпуса;
    \item подбор дикторского состава;
    \item запись речевых фрагментов, произнесенных дикторами;
    \item проверка качества записи речевых фрагментов;
    \item фонетическая верификация речевых фрагментов и их разметка;
    \item обработка результатов верификации;
    \item окончательное формирование речевого корпуса.
\end{itemize}
\subsection{Основные проблемы}
На каждом этапе возникают определенные проблемы.
\begin{itemize}
    \item Формирование стандартов записи звуков;
    \item в~зависимости от~целей исследования необходимо подобрать
    наиболее оптимальный текст: труднопроизносимые слова и слишком длинные
    предложения могут быть записаны с~ошибками;
    \item запись надо очистить от~шумов и проверить на~оговорки;
    \item аннотирование и~т.п.
\end{itemize}

\subsection{Отбор информантов}
В зависимости от~задач подбираются те или иные информанты. Например,
для~исследования определенного акцента, произношения или дефекта,
требуются одни носители языка, а для исследования норм или для~распознавания
речи~--- иные. Учитываются такие характеристики потенциальных информантов, как
\begin{itemize}
	\item возраст
	\item пол
	\item образование
	\item место проживания
	\item профессия
	\item \dots
\end{itemize}

\subsection{Фонетические базы данных}
Имеется несколько речевых баз данных для~русского языка, среди которых
ISABASE, МУРКО, УМКО, РЭК, <<Один речевой день>> и другие.

\subsection{Способы преобразования звучащей речи в~текст}
В~зависимости от~объемов и задач расшифоровка бывает автоматическая,
полуавтоматическая и полностью ручная. Среди автоматических парсеров
используют Praat и ELAN. Они позволяют выполнять визуализацию аудиофайлов (спектральный анализ и т.д.), создавать многослойную разметку, синтезировать звук по спектральным характеристикам и т.д. Система Praat также поволяет обучать собственные алгоритмы распознавания речи с использованием специального скриптового языка.

\subsection{Структура и организация речевых баз данных}
Большинство крупномасштабных речевых корпусов формируется с~целью их
использования для~решения задач распознавания речи, и поэтому они обычно
содержат разделы, предназначенные для~обучения систем распознавания и
для~последующего тестирования качества работы этих систем.

Каждый из~этих двух разделов обычно еще разбивается на~подразделы, каждый
из~которых, в~свою очередь, содержит речевые фрагменты, произнесенные
одним диктором. При~этом желательно поддерживать выполнение двух требований.
Во-первых, не должны пересекаться составы дикторов, относящихся к~этим
разделам, а, во-вторых, наборы предложений, произнесенных дикторами,
относящимися к~разделу обучения и к~разделу тестирования, должны быть
различны.

Иногда бывает удобно разделы обучения и тестирования структурировать
дополнительно, разбивая их на~подразделы, соответствующие, скажем,
дикторам-мужчинам и дикторам-женщинам, или на~подразделы, соответствующие
различным диалектам дикторов.

Бывает целесообразно также, наряду с~основными разделами обучения и
тестирования формировать еще один специальный раздел речевой базы,
предназначенный для~целей отладки и совершенствования самой системы
распознавания речи. В~этом случае желательно, чтобы состав дикторов,
соответствующих этому разделу, не~пересекался с~составами дикторов
для~разделов обучения и тестирования. Это ограничение накладывается
также и на~множество предложений, подготовленных для~такого раздела.

Если формируемая речевая база предназначена не~только для~использования
в~системах автоматического распознавания речи, но~и для~теоретических
исследований в~области фонетики, то желательно структурировать ее еще
и в~соответствии с~фонетическими особенностями лексического материала,
такими, например, как фонетическая полнота, фонетическая сбалансированность,
фонетическая репрезентативность и~т.п. В~этом случае в~те или иные разделы
базы заносятся речевые фрагменты, образованные дикторами при~произнесении
наборов предложений, обладающих соответствующими фонетическими особенностями.

\subsection{Применение звуковых баз данных}
Позволяет тестировать и разрабатывать методы распознавания речи,
изучать лингвистическую динамику записанного материала:
исследовать временные ряды количественных переменных с~помощью
стандартных статистических методов и анализировать частотные ряды
(лексики, грамматических и, в~частности, синтаксических структур,
семантики или разговорных тем, тех или иных акустических явлений или
просодических контуров) в~зависимости от~времени суток и условий
коммуникации в~самом широком понимании этого термина, а также решает
множество других задач, таких как анализ влияния профессии на~бытовую
жизнь человека, получение информации о среднем артикуляционном темпе
спонтанной речи носителей русского языка.

\section{Корпусы звучащей речи для русского языка; корпус <<Один речевой день>>,
аннотирование корпуса, программное обеспечение}
В~этом разделе мы продолжим обсуждать речевые базы данных. Корпус
<<Один речевой день>> исследует спонтанную человеческую речь.
Производится отбор анонимных информантов, незнакомых группе
исследователей. Им крепится диктофон и далее, в~течение дня,
производится запись всех разговоров. Затем исследователь производит
расшифровку записи и описывает сцены.

Аннотирование производится так. Во-первых, заполняется ряд таблиц
об~информаторе и участвующих в~записи лиц. Каждой реплике
задается тайм-код, ее расшифровка и некоторые специальные
символы: обозначение пауз, вопросов и восклицаний, растягивание
гласных и~т.п.

Хранение информации производится в~MS~Access.

\section{Моделирование в лингвистике;  свойства лингвистических моделей}
Современная структурная лингвистика~--- по~большей части, наука
о~моделях языка. Модель~--- формализованное описание объекта,
системы объектов, выраженное конечным набором предложений какого-либо
языка, формулами, таблицами, графиками и~т.д.

\paragraph{Свойства лингвистических моделей.}
\begin{enumerate}
	\item Моделировать можно только такие явления, существенные
    свойства которых исчерпываются их структурными характеристиками
    и никак не~связаны с~их физической природой. Моделью такого объекта
    должно считаться любое устройство, функционально похожее на~него
    (функциональная аппроксимация объекта).
	\item Модель всегда является некоторой идеализацией объекта
    (например, принцип подстановки недостающих членов предложения
    в~синтаксисе; предположение о~бесконечном числе возможных правильных
    предложений языка).
	\item Модель оперирует не~понятиями о~реальных объектах, а конструктами,
    понятиями об~идеальных объектах (например, конструктом является
    понятие нулевой флексии).
	\item Всякая модель должна быть формальной~--- явно и однозначно
    заданы исходные объекты, связывающие их утверждения и правила
    обращения с~ними.
	\item Модель должна обладать объяснительной силой, то есть объясняет
    факты, необъяснимые старой теорией, и предсказывает неизвестное раньше,
    но возможное поведение объекта.
\end{enumerate}

\paragraph{Построение модели.}
Основные шаги: фиксирование фактов, требующих объяснения; выдвижение гипотез
для~объяснения фактов; реалиация гипотез в~виде моделей; экспериментальная
проверка модели.

Типы лингвистических моделей:
\begin{itemize}
	\item модели, имитирующие речевую деятельность человека (моделируем
    конкретные языковые процессы и явления);
	\item модели исследования (моделируем исследовательскую деятельность
    лингвиста);
	\item метатеории.
\end{itemize}

Кроме того, выделяются аналитические, синтетические и порождающие модели.

\section{Представление письменного текста в компьютерных системах; кодировки операционных систем; UNICODE}
\begin{defn}
\textsl{Письмо}~--- знаковая система фиксации речи, позволяющая с~помощью
начертательных (графических) элементов передавать речевую информацию
на~расстоянии и закреплять её во~времени. Существует 4 основных типа
письма~--- идеографический, словесно-слоговой, собственно силлабический
и буквенно-звуковой (алфавитный).

{\sl Набор символов} (англ. {\sl character set})~--- таблица, задающая
кодировку конечного множества символов элементов текста: букв, цифр,
знаков препинания и др.

{\sl Кодовая страница}~--- набор одновременно применяемых цифровых символов,
каждому из~которых соответствует цифровой код.
\end{defn}

Кодировкой называют как стандарт кодирования, так и соответствующий набор
символов.

\subsection{Кодировки}
До~середины девяностых годов в~основном использовался стандарт \textbf{ASCII}
(American Standard Code for Information Interchange), который нормирует систему
кодирования букв английского алфавита, цифр от~0 до~9, знаков препинания, а также
служебных (управляющих) символов. Для~языков с~латинским алфавитом была создана
расширенная таблица, в~которой некоторые служебные знаки заменены на~буквы
национальных алфавитов, чаще всего буквы с~диакритическими знаками. Для~языков,
в~которых используются другие алфавиты, применялись кодовые страницы, в~которых
половину кодовой таблицы (коды~0--127) занимают буквы латинского алфавита,
а другую половину~(128--255)~--- дополнительные символы, в~частности буквы любого
другого алфавита. Таким образом, кодовая страница содержала максимум 256 символов,
включая управляющие символы. Такое число кодов позволяло представить два алфавита,
например, английский и русский, однако при~этом представление других знаков
было невозможным. В~языках разметки (html, xml) в~настоящее время для~записи
специальных символов и букв с~диакритическими знаками применяются идентификаторы,
например, \texttt{agrave}, \texttt{aacute}.

В~1991~г. некоммерческой организацией Unicode Consortium, Unicode предложен стандарт
\textbf{Unicode} (Юникод), который содержит коды для~разных систем письма.
В~первой версии Unicode, которую называют \textbf{UTF\nobreakdash-8} (Unicode
Transformation Format) принята кодировка с~фиксированным размером символа в~16 бит,
т.е. общее число кодов было~$2^{16}$ ($65\;536$ символов). Существует практика
обозначения кодов символов числами в~шестнадцатеричной системе (например, U+04F0).
Во~второй версии UTF\nobreakdash-16 (1996~г.) кодовая область значительно расширена.
В~UTF\nobreakdash-16 можно отобразить $1\;112\;064$ символов, что практически полностью
охватывает все современные и исторические системы письма, а также математическую
и музыкальную нотации. В~настоящее время Unicode является наиболее распространенным
стандартом, но и стандарт ASCII не~вышел из~употребления.

\subsection{UNICODE}
Стандарт состоит из~двух основных разделов: \textsl{универсальный набор символов}
(англ. UCS, universal character set) и \textsl{семейство кодировок} (англ.~UTF,
Unicode transformation format). Универсальный набор символов задаёт однозначное
соответствие символов кодам~--- элементам кодового пространства, представляющим
неотрицательные целые числа. Семейство кодировок определяет машинное представление
последовательности кодов UCS.

Коды в~стандарте Юникод разделены на~несколько областей. Область с~кодами от~U+0000
до~U+007F содержит символы набора ASCII с~соответствующими кодами.
Далее расположены области знаков различных письменностей, знаки пунктуации и
технические символы. Часть кодов зарезервирована для~использования в~будущем.

\section{Транслитерация и транскрипция; проблема различного написания слов; способы ее решения}
\begin{defn}
\textsl{Транслитерация}~--- конверсия систем письма, при~которой каждый
графический элемент (знак) одной системы письма представляется (заменяется)
одним и тем~же графическим элементом другой системы письма.

\textsl{Ослабленная транслитерация}~--- замена некоторых букв исходного текста
сочетанием двух или более букв чужого алфавита.

\textsl{Строгая транслитерация}~--- замена каждой буквы исходного текста
только одной буквой другой письменности.

\textsl{Транскрипция}~--- способ однозначной фиксации на~письме звуковых
отрезков речи.
\end{defn}

\subsection{Проблема различного написания слов и способы ее решения.}
Чаще всего разночтения встречаются в~написании собственных имен. Некоторые
из~них (изменения имен, псевдонимы, переименования географических объектов и~т.д.)
мало имеют отношения собственно к~компьютерной лингвистике, хотя чрезвычайно
существенны при создании информационно-поисковых языков.

Для~унификации транслитерации создан стандарт \textbf{ISO~9:95}. В~нем
описаны 2~системы транслитерации: система А для~строгой транслитерации
и система В для~ослабленной.

Для~представления иностранных имен используется не~только транслитерация,
но и транскрипция, т.е. фиксация на~письме звучания слова. В~реальности
эти два подхода сосуществуют, а порой смешиваются.

Передача на~русском языке имен Ивлин Во (Evelyn Waugh) или Шарль Бодлер
(Charles Baudelaire) является примером транскрипции. С~другой стороны,
при~передаче на~русском языке имени немецкого поэта Генриха Гейне (Heinrich
Heine) использовалась транслитерация.

\section{Графематический анализ; токенизация; основные проблемы ГА}
\textsl{Графематический анализ} текста состоит в~следующем:
\begin{enumerate}
    \item разделение входного текста на~элементы (слова,
    разделители и~т.п.);
    \item удаление нетекстовых элементов;
    \item выделение и оформление нестандартных элементов, например:
    \begin{itemize}
        \item элементов форматирования~--- жирность, курсивность,
        подчеркивание;
        \item структурных элементов текста~--- заголовков, абзацев,
        примечаний;
        \item различных элементов текста, не~являющихся словами
        (числа, даты в~цифровых форматах, буквенно-цифровые комплексы
        и~т.п.);
        \item имен~(ИО), написанных инициалами;
        \item иностранных лексем, записанных латиницей и~т.п.
    \end{itemize}
\end{enumerate}

Cложности:
\begin{itemize}
    \item обработка дефиса и пробела;
    \item выделение составных предлогов, устойчивых оборотов, аналитических форм и~др.;
    \item иноязычные фрагменты;
    \item нетекстовые элементы.
\end{itemize}

\section{Форматы представления данных в лингвистике; форматы структуры (позиционные, с ключевыми словами, со справочником ISO 2709)}
\subsection{Форматы представления данных в лингвистике.}
\begin{itemize}
    \item Корпуса
    \begin{itemize}
        \item XML
        \item TEI
    \end{itemize}
    \item Онтологии
    \begin{itemize}
        \item RDF
        \item OWL
        \item \dots
    \end{itemize}
\end{itemize}

\subsection{Форматы структуры}
% нужны примеры!!!
\begin{itemize}
    \item позиционные форматы
    \item форматы с~ключевыми словами
    \item форматы со~справочником
\end{itemize}

\subsection{ISO 2709}
\label{subsec:ISO}
Стандарт \textbf{ISO 2709}~--- это универсальный стандарт
библиографического описания, предназначенный для~передачи
библиографической информации. Запись в~этом формате предполагает
наличие трех полей:
\begin{enumerate}
    \item маркер записи;
    \item справочник~--- содержит метки для~каждого поля
    (индекс поля, начальная позиция данных, длина содержимого поля);
    \item поля данных переменной длины;
\end{enumerate}

\textbf{Поля, обязательные для всех записей}
\\
001 ИДЕНТИФИКАТОР ЗАПИСИ \\
100 ДАННЫЕ ОБЩЕЙ ОБРАБОТКИ \\
200 ЗАГЛАВИЕ И СВЕДЕНИЯ ОБ ОТВЕТСТВЕННОСТИ (подполе \$a обязательно для каждой записи) \\
801 ИСТОЧНИК ЗАПИСИ \\

\section{Языки разметки (SGML, HTML, XML), их особенности}
\begin{defn}
{\sl Язык разметки}~--- набор символов или последовательностей,
вставляемых в~текст для~передачи информации о~его выводе или строении.
Принадлежит классу компьютерных языков. Текстовый документ, написанный
с~использованием языка разметки, содержит не~только сам текст (как
последовательность слов и знаков препинания), но и дополнительную
информацию о~различных его участках~--- например, указание на~заголовки,
выделения, списки и~т.д. В~более сложных случаях язык разметки
позволяет вставлять в~документ интерактивные элементы и содержание
других документов.
\end{defn}
SGML, HTML, XML~--- теговые языки разметки документов. Любой документ
представляет собой набор элементов, причём начало и конец каждого
элемента обозначается специальными пометками~--- тегами. Элементы
могут быть пустыми, то~есть не~содержащими никакого текста и других
данных (например, тег перевода строки \texttt{<br>}). Кроме того,
элементы могут иметь атрибуты, определяющие какие\nobreakdash-либо
их свойства (например, размер шрифта для элемента \texttt{font}).
Атрибуты указываются в~открывающем теге.

\paragraph{SGML.}
Основные части документа SGML:
\begin{enumerate}
	\item SGML-декларация~--- определяет, какие символы и ограничители
    могут появляться в~приложении;
	\item Document Type Definition~--- определяет синтаксис конструкций
    разметки. DTD может включать дополнительные определения, такие,
    как символьные ссылки\nobreakdash-мнемоники;
	\item Спецификация семантики, относится к~разметке~--- также
    даёт ограничения синтаксиса, которые не~могут быть выражены
    внутри DTD;
	\item Содержимое SGML\nobreakdash-документа~--- по~крайней мере,
    должен быть корневой элемент.
\end{enumerate}

\paragraph{HTML.}
HTML является приложением SGML, а это означает, что все элементы языка
и правила формирования структуры документов четко определены в~соответствующих
DTD (HTML~2.0, HTML~3.2, HTML~4.0). Это стало причиной основного недостатка
HTML~--- его ограниченности. В HTML~DTD определен фиксированный набор
допустимых элементов разметки и правил их расстановки. Расширение этого
набора возможно только при~принятии очередной версии стандарта, а
это~--- чрезвычайно долгий и трудный процесс, осложняющийся постоянными
разногласиями крупнейших разработчиков html-браузеров (программ просмотра)
по поводу полной или частичной поддержки нового стандарта, или введения
собственных расширений языка, идущих вразрез со`стандартизированными
соглашениями. Таким образом, чем более специализированной становится
область применения HTML (например, публикация инженерных расчетов,
изобилующих сложными формулами), тем сильнее ощущаются ограничения HTML.
Кроме того, теги HTML определяют только способ отображения элементов,
а не~их значение.

\paragraph{XML.} Все составляющие части документа обобщаются в~пролог
и корневой элемент. Корневой элемент~--- обязательная часть документа,
составляющая всю его суть(пролог, вообще говоря, может отсутствовать).
Может включать (а может не~включать) вложенные в~него элементы и
символьные данные, а также комментарии. Вложенные в~корневой элемент
элементы, в~свою очередь, могут включать вложенные в~них элементы,
символьные данные и комментарии, и так далее. Пролог может включать
объявления, инструкции обработки, комментарии. Его следует начинать
с~объявления XML, хотя в~определённой ситуации допускается отсутствие
этого объявления.

Элементы документа должны быть правильно вложены: любой элемент,
начинающийся внутри другого элемента (то~есть любой элемент документа,
кроме корневого), должен заканчиваться внутри элемента, в~котором
он начался. Символьные данные могут встречаться внутри элементов
как непосредственно так и в~специальных секциях «CDATA». Объявления,
инструкции обработки и элементы могут иметь связанные с~ними атрибуты.
Атрибуты используются для~связывания с~логической единицей текста
пар имя-значение.

\paragraph{Особенности.}
\begin{itemize}
	\item \textbf{SGML} не~разрешает опускать закрывающие теги.
	\item Теги \textbf{XML} в~отличие от HTML и SGML не~определяют
    способа представления данных, а лишь задают структурную организацию.
	\item В~отличие от~HTML, язык \textbf{XML} не~определяет
    фиксированного набора управляющих элементов, а позволяет
    разработчику самостоятельно определить любой набор тэгов, как это
    возможно в SGML. Этим устраняется такой недостаток HTML, как
    ограниченность формата.
\end{itemize}

\section{Форматы наполнения; UNIMARC}
Формат MARC (Machine Readable Cataloging) был разработан Библиотекой
Конгресса США в 1965\nobreakdash-1966 годах с~целью получения данных
каталогизации в машиночитаемой форме. Аналогичная работа выполнялась
в~Великобритании Советом по~Британской национальной каталогизации
для~обеспечения использования машиночитаемых данных при~подготовке
печатного издания Британской национальной библиографии~--- British
national Bibliography (проект BNB MARC). На~основе указанных разработок
в~1968~г. начал создаваться коммуникативный англо\nobreakdash-американский
формат MARC (проект MARC~II). Целями его создания стало обеспечение:
гибкости решения каталогизационных и других библиотечных задач и пригодности
для~национального библиографического описания любых видов документов
и использования структуры записи в~автоматизированных системах.
Позднее USMARC превратился в~комплекс специализированных форматов
(USMARC Concise Formats) для~записи библиографических, авторитетных,
классификационных данных, данных о фондах и общественной информации,
которые были с~трудом совместимы.

Для~преодоления несовместимости был разработан формат UNIMARC
(Universal Machine Readable Cataloging). Предполагалось, что этот
формат должен стать посредником между любыми национальными версиями
форматов MARC и, следовательно, обеспечивать конвертирование данных
из~национального формата в~него, а из~него~--- в~другой национальный
формат.

Формат UNIMARC (MARC21) следует стандарту ISO~2709
(см.~\hyperref[subsec:ISO]{соответствующий вопрос}).

\section{Форматы наполнения; рекомендации TEI}
\label{sec:TEI}
Кроме форматов структуры, которые описывают правила представления
структуры, существуют стандарты, регламентирующие заполнение этой
структуры,~--- форматы наполнения.

Система кодирования текстов (TEI) направлена на~обеспечение обмена
информацией, хранимой в~электронной форме. Основное внимание
уделяется текстовой информации, но предусмотрены средства и для~других
форм, например, для~графических изображений и звуковой информации
(всего около двадцати). Все тексты в формате TEI содержат заголовок
TEI (размечаемый как элемент \texttt{<teiHeader>}) и собственно текст
(размечаемый как элемент \texttt{<text>}). Заголовок \texttt{teiHeader}
должен содержать следующие элементы:
\begin{itemize}
    \item \texttt{<fileDesc>}~--- полное библиографическое описание файла;
    \item \texttt{<encodingDesc>}~--- описывает отношение между текстом
    и его источником;
    \item \texttt{<profileDesc>}~--- металингвистическое описание текста
    (язык, ситуация, в~которой текст был создан и~т.д.);
    \item \texttt{<revisionDesc>}~--- история правок файла.
\end{itemize}
Элементы, которые размечаются в~документах всех типов:
\begin{itemize}
    \item абзацы
    \item знаки препинания
    \item выделения и цитирование
    \item редакторские правки
    \item имена, адреса и другие именованные сущности
    \item рисунки и прочие нетекстовые элементы
    \item списки
    \item примечания
    \item ссылки
\end{itemize}
Текст TEI может быть монолитным (отдельное произведение) или
объединенным (набор отдельных произведений, \texttt{teiCorpus}).
В~любом случае текст может иметь необязательные вводную часть
и закрывающую часть. Между ними располагается основная часть
текста, которая, в~случае объединенного текста, может состоять
из~групп, а они, в~свою очередь, из~других групп или текстов.
Простой документ~TEI на~текстовом уровне состоит из~следующих
элементов:
\begin{itemize}
    \item \texttt{<front>} содержит различную вступительную
    информацию (заголовки, титульный лист, предисловия, посвящения
    и~т.п.), которую размещают перед основным текстом;
    \item \texttt{<group>} содержит набор монолитных текстов
    или групп текстов;
    \item \texttt{<body>} содержит всю основную часть одного
    монолитного текста, исключая то, что относится к~вводной
    или закрывающей частям текста;
    \item \texttt{<back>} содержит различные приложения и~т.п.,
    которые располагаются после основной части текста.
\end{itemize}
Как отдельные специфические виды текстов размечаются стихотворные
и драматические произведения, а также транскрипции звучащей речи.

\section{Разметка корпусов, в том числе в TEI}

Под аннотацией (разметкой) корпуса понимается процесс и результат приписывания входящим в корпус текстам и их компонентам некоторой метаинформации: это могут быть как данные об авторе текста, дате публикации (экстралингвистическая разметка), так и информация о членении текста на предложения и абзацы и лингвистические характеристики отдельных элементов (предложений, слов, морфем)~--- лингвистическая разметка.

\paragraph{Морфологическая разметка.} Она касается уровня словоформ: каждой словоформе приписывается её лексема с частеречным показателем, а также набор граммем, причем отдельная граммема или полная грамматическая характеристика слова, называются тегом (этот термин пришел из терминологии языков разметки, таких как XML и HTML, в связи с тем, что большинство корпусов размечаются именно с помощью таких технологий). Стоит также отметить, что в работах по автоматической обработке текста часто вместо терминов «словоформа» и «лексема» употребляют термины «токен» и «лемма» соответственно . В иностранной терминологии употребляется термин part-of-speech tagging (POS-tagging), дословно~--- частеречная разметка. В действительности морфологические теги включают не только признак части речи, но и признаки грамматических категорий, свойственных данной части речи. Для каждой словоформы определяется одна или несколько лемм и набор грамматических признаков этой словоформы. Совокупность грамматических признаков конкретной словоформы называются её морфологическим разбором.

Первый корпус английских текстов, Brown Corpus, появился в середине 1960-х годов. Корпус насчитывал около миллиона словоупотреблений и размечался на протяжении долгого времени. В этом корпусе используется расширенная система частеречных тегов: существительные, например, могут быть помечены как NN, NNS или NP (нарицательные в единственном числе, нарицательные во множественном числе и собственные в единственном числе соответственно). Первая программа разметки состояла из описания правил совместного употребления слов (например, слово после артикля могло быть существительным, но не могло быть глаголом) и достигала семидесятипроцентной точности. В дальнейшем разметка исправлялась вручную. Система тегов Брауновского корпуса (Brown corpus tagset)  является одной из наиболее распространенных для английского языка, наряду с тегсетом корпуса Penn Treebank . Набор тегов Penn Treebank представляет собой сокращенную версию набора из Брауновского корпуса, с его помощью размечены многие корпуса современного английского языка.

\paragraph{Синтаксическая разметка.} Синтаксическая разметка является результатом парсинга, выполняемого на основе данных морфологического анализа. Этот вид разметки зависит от принятой формальной синтаксической модели и описывает синтаксические связи между лексическими единицами и/или различные синтаксические конструкции (например, придаточное предложение, именное сказуемое и т. д.). В отличие от морфологии, способы представления синтаксической структуры и синтаксических отношений не столь унифицированы.

В результате работы программ автоматической синтаксической разметки фиксируются синтаксические связи между словами и словосочетаниями, а синтаксическим единицам приписываются соответствующие характеристики (тип предложения, синтаксическая функция словосочетания и т. п.).
Однако автоматический анализ естественного языка небезошибочен и неоднозначен – он, как правило, дает несколько вариантов анализа для одной и той же языковой единицы (слова, словосочетания, предложения). При создании корпусов для снятия неоднозначности используются
автоматический и ручной способы обработки. Корпусы нового поколения включают сотни миллионов слов, поэтому требуются системы, которые бы минимизировали вмешательство человека. Автоматическое разрешение морфологической или синтаксической неоднозначности, как правило, основывается на учете
контекста и использовании информации более высокого уровня (синтаксического, семантического) с применением статистических методов.

\paragraph{Семантическая разметка.} Семантические теги чаще всего обозначают семантические категории, к которым относится данное слово или словосочетание, и более узкие подкатегории, специфицирующие его значение. Семантическая разметка корпусов предусматривает спе цификацию значения слов, разрешение омонимии и синонимии, категоризацию слов (разряды), выделение тематических классов, признаков каузативности, оценочных и деривационных характеристик и т.д.

Существуют и другие типы разметки, в частности:
\begin{itemize}
\item анафорическая разметка. Она фиксирует референтные связи, например, местоименные;
\item просодическая разметка. В просодических корпусах применяются теги, обозначающие ударение и интонацию. В корпусах устной разговорной речи просодическая разметка часто сопровождается так называемой дискурсной разметкой, которая служит для обозначения пауз, повторов, оговорок и т.д.
\end{itemize}

Про TEI см. \hyperref[sec:TEI]{предыдущий вопрос}?

\section{Семантический уровень анализа языка; семантические проблемы в~традиционной лингвистике}
\textsl{Семантика}~--- раздел лингвистики (в частности, семиотики),
изучающий смысловое значение единиц языка. В~качестве инструмента
изучения применяют семантический анализ.

Непосредственно наблюдаемая ячейка семантики~--- полнозначное слово
(например, существительное, глагол, наречие, прилагательное)~---
организована по~принципу «семантического треугольника»: внешний
элемент~--- последовательность звуков или графических знаков
(означающее)~--- связан в~сознании и в~системе языка, с~одной стороны,
с~предметом действительности (вещью, явлением, процессом, признаком),
называемым в~теории семантики {\sl денотатом}, референтом, с~другой стороны~---
с~понятием или представлением об~этом предмете, называемым смыслом,
{\sl сигнификатом}, интенсионалом, означаемым.

Другой универсальной ячейкой семантики является предложение (высказывание),
в~котором также выделяются денотат (или референт) как обозначение факта
действительности и сигнификат (или смысл), соответствующий суждению об~этом
факте. Денотат и сигнификат в~этом смысле относятся к~предложению в~целом.
В~отношении~же частей предложения обычно подлежащее (или субъект) денотатно,
референтно, а сказуемое (или предикат) сигнификатно.

Аналогично слову и предложению организована семантика всех единиц языка.
Она распадается на~две сферы~--- предметную, или денотатную (экстенсиональную),
семантику и сферу понятий, или смыслов,~--- сигнификатную (интенсиональную)
семантику. Термины <<экстенсиональная семантика>> и <<интенсиональная семантика>>
восходят к~описанию отдельного слова-понятия, где ещё в традиции средневековой
логики объём понятия (т.е. объём его приложений к~предметам, покрываемая
предметная область) назывался термином extensio ‘растяжение’, а содержание
понятия (т.е. совокупность мыслимых при~этом признаков)~--- словом intensio
‘внутреннее натяжение’. Денотатная и сигнификатная сферы семантики в~естественных
языках (в отличие от некоторых специальных искусственных языков) строятся
довольно симметрично, при~этом сигнификатная (понятийная) в~значительной
степени копирует в~своей структуре денотатную (предметную) сферу. Однако
полный параллелизм между ними отсутствует, и ряд ключевых проблем семантики
получает решение только применительно к~каждой сфере в~отдельности. Так,
предметная, или денотатная, синонимия, экстенсиональное тождество языковых
выражений не~обязательно влекут за~собой сигнификатную, или понятийную,
синонимию, интенсиональное тождество, и наоборот.

Семантические отношения описываются семантикой как разделом языкознания
с~разных точек зрения. К~{\sl парадигматике} относятся группировки слов в~системе
языка, основой которых выступает оппозиция,~--- синонимия, антонимия, гипонимия,
паронимия, гнездо слов, семья слов, лексико-семантическая группа, а также
наиболее общая группировка слов~--- поле. Различаются поля двух основных
видов:
\begin{enumerate}
    \item объединения слов по их отношению к~одной предметной области~---
    предметные, или денотатные, поля, например цвето-обозначения, имена
    растений, животных, мер и весов, времени и~т.д.;
    \item объединения слов по~их отношению к~одной сфере представлений или
    понятий~--- понятийные, или сигнификатные, поля, например обозначения
    состояний духа (чувств радости, горя, долга), процессов мышления,
    восприятия (видения, обоняния, слуха, осязания), возможности, необходимости
    и~т.п.
\end{enumerate}
В~предметных полях слова организованы преимущественно по~принципу <<пространство>>
и по~принципам соотношения вещей: часть и целое, функция (назначение) и ее
аргументы (производитель, агенс, инструмент, результат); в~понятийных полях~---
преимущественно по принципу <<время>> и по~принципам соотношения понятий
(подчинение, гипонимия, антонимия и др.). Парадигматические отношения
формализуются с~помощью математической теории множеств.

К~{\sl синтагматике} относят группировки слов по~их расположению в~речи
относительно друг друга (сочетаемость, аранжировка). Основой этих отношений
выступает {\sl дистрибуция} (см.~Дистрибутивный анализ). Они формализуются
с~помощью математической теории вероятностей, статистико-вероятностного
подхода, исчисления предикатов и исчисления высказываний, теории алгоритмов.

При~соотнесении результатов описания семантики в~парадигматике и синтагматике
выявляются некоторые их общие черты, наличие семантических инвариантов,
а также более мелкие и более универсальные, чем слово, семантические
единицы~--- семантические признаки, или семы (называемые также компонентом,
иногда семантическим параметром или функцией). Основные семы в~лексике совпадают
с~категориальными грамматическими значениями в~грамматике (граммемы).
В~парадигматике сема выявляется как минимальный признак оппозиции,
а в~синтагматике~--- как минимальный признак сочетаемости.

Одна из~важнейших проблем семантики~--- системность лексических значений~---
имеет доступ к~своему исследованию со~стороны синтаксиса, что позволяет
объективным путем, посредством использования внутриязыковых критериев,
обнаруживать те семантические связи между словами, с~помощью которых данные
слова образуют лексические подсистемы, поля, группы, т.е. совокупности
слов, имеющие семантическую общность. Так, например, лишь существительные
типа вид, семейство, класс, разряд, категория, группа, разновидность, род
и~т.п. допускают следующую трансформацию: этот вид (семейство, класс, разряд
и.~т.п.)~--- объекты этого вида (семейства, класса, разряда и т.п.); только
существительные со~значением вместилища допускают трансформацию типа {\it банка
из-под~варенья}~--- {\it банка под~варенье}, {\it бутылка из-под~молока}~---
{\it бутылка под~молоко} и~т.п.; только существительные со~значением параметра вещей
допускают трансформацию типа {\it высотой с~дом}~--- {\it высокий, как дом},
{\it шириной с~улицу}~--- {\it широкий, как улица} и~т.п. Поскольку наиболее
активными с~синтаксической точки зрения являются глаголы, исследование их
семантики служит удачной сферой приложения дистрибутивно-трансформационного
метода с~целью, например, таксономии глаголов с~семантической точки
зрения, что и осуществлено на~материале русского языка.

\section{Компьютерная семантика. Ее отличия от традиционной лингвистической.}
Семантический (смысловой) анализ текста~--- одна из~ключевых проблем
как теории создания систем искусственного интеллекта, относящаяся
к~обработке естественного языка, так и компьютерной лингвистики.
Результаты семантического анализа могут применяться для~решения задач
в~таких областях как, например, психиатрия (для диагностирования
больных), политология (предсказание результатов выборов), торговля
(анализ <<востребованности>> тех или иных товаров на~основе комментариев
к~данному товару), филология (анализ авторских текстов), поисковые
системы, системы автоматического перевода и~т.д.

Несмотря на~свою востребованность практически во~всех областях жизни
человека, семантический анализ является одной из~сложнейших математических
задач. Вся сложность заключается в~том, чтобы <<научить>> компьютер
правильно трактовать образы, которые автор текста пытается передать
своим читателям/слушателям.

Способность <<распознавать>> образы считается основным свойством
человеческих существ, как, впрочем, и других живых организмов.
Образ представляет собой описание объекта. В~каждое мгновение
нашего бодрствования мы совершаем акты распознавания. Мы опознаем
окружающие нас объекты и в~соответствии с~этим перемещаемся и
совершаем определенные действия.

Естественный язык в~отличие, например, от~компьютерных (алгоритмических)
языков формировался во~многом стихийно, не~формализовано. Это
обуславливает целый ряд сложностей в~понимании текста, вызванных,
например, неоднозначным толкованием одних и тех~же слов в~зависимости
от~контекста, который может быть и неизвлекаем, в~принципе, из~самого
текста. Следовательно, этот контекст или знание о~предметной области
в~систему должны быть заранее внесены. К~тому~же зачастую практические
задачи требуют точного определения времени, места того, что описано
в~тексте, точной идентификации людей и~т.д., в~то время как подобная
информация находится за~пределами данного текста. В~этом случае система
может или не~обрабатывать эту информацию, или оставить ее до~выяснения
контекста и даже попытаться проявить инициативу в~выяснении контекста,
например, в~диалоге с~оператором, задающим ввод текста. То, как ведет
себя система в~подобной ситуации, определяется стилем и схемой работы
системы.

Промышленные системы автоматической обработки текста, в~основном, сейчас
используют два этапа анализа текста: морфологический и синтаксический.
Однако теоретические разработки многих исследователей предполагают
существование следующего за~синтаксическим этапа~--- семантического.
В~отличие от~предыдущих шагов семантический этап использует формальное
представление смысла составляющих входной текст слов и конструкций.
Суть семантического анализа понимается разными исследователями по-разному.
Многие ученые сходятся во~мнении, что в~сферу семантического анализа входит:
\begin{itemize}
    \item построение семантической интерпретации слов и конструкций;
    \item установление <<содержательных>> семантических отношений
    между элементами текста, которые уже принципиально не~ограничены
    размером одного слова.
\end{itemize}

Основные проблемы понимания текста в~обработке естественных языков таковы:
\begin{enumerate}
    \item Знание системой контекста и проблемной области и обучение
    этому системы. Например, из~предложения <<мужчина вошел в~дом
    с~красным портфелем>> можно извлечь как представление о~мужчине
    с~красным портфелем, так и о~доме с~красным портфелем, если
    заранее не~иметь в~виду, что применительно к~мужчинам употребление
    принадлежности портфеля гораздо вероятнее, чем применительно к~дому.
    \item Различная форма передачи синтаксиса (т.е. структуры) предложения
    в~разных языках. Например, если синтаксическая роль слова (подлежащее,
    сказуемое, определение и~т.д.) в~английской речи во~многом определяется
    положением слова в~предложении относительно других слов, то в~русском
    предложении существует свободный порядок слов и для~выявления
    синтаксической роли слова служат его морфологические признаки
    (например, окончания слов), служебные слова и знаки препинания.
    \item Проблема равнозначности. Предложения <<длинноухий грызун
    бросился от~меня наутек>> и <<заяц бросился от~меня наутек>>
    могут означать одно и то~же, но могут иметь и разный смысл,
    например, если в~первом случае имелся в~виду длинноухий
    тушканчик.
    \item Наличие в~тексте новых для~компьютера слов, например неологизмов.
    Самообучаемая система должна уметь <<интуитивно>> определить
    (возможно, и неправильно, но с~возможность в~дальнейшем исправить себя)
    лексическую роль, морфологическую форму этого слова, попробовать вписать
    его в~существующую структуру знаний, наделить его какими-то атрибутами
    или выяснить все это в~диалоге с~оператором. Система, не~способная
    к~самообучению просто потеряет какое-то количество информации.
    \item Проблема совместимости новой информации с~уже накопленными
    знаниями. Новая информация может каким-то образом противоречить
    уже накопленной информации. Необходимо реализовать механизм, определяющий,
    в~каких случаях нужно отвергнуть старую информацию, а в~каких~--- новую.
    \item Проблема временных противоречий. Так в~предложении <<я думал,
    что сверну горы>> глагол в~прошедшей форме <<думал>> сочетается
    с~глаголом будущего времени <<сверну>>.
    \item Проблема эллипсов, то есть предложений с~пропущенными фактически,
    но существующими неявно благодаря контексту словами. Например,
    в~предложении <<я передам пакет тебе, а ты~--- Ивану Петровичу>>
    во~второй части опущен глагол <<передашь>> и существительное <<пакет>>.
\end{enumerate}

Лингвистический процессор может быть интегрирован с~системой распознавания
и (или) синтеза речи, что может сделать процесс общения с~компьютером
максимально удобным, а, следовательно, и продуктивным.

Одной из~наиболее очевидных направлений применения лингвистических
процессоров является машинный перевод с~одного естественного языка (ЕЯ)
на~другой.

Системы семантического анализа не~могут существовать без~морфологической
составляющей. В~качестве морфологической составляющей выступают различные
виды словарей словоформ (т.е. содержащие все варианты склонения, спряжения
и~т.д. того или иного слова). Но возникает проблема <<неполноты>> того или
иного словаря. Существует ряд подходов для~решения этой проблемы.

Системы семантического анализа не~могут существовать и без~синтаксической
составляющей. Основной задачей синтаксического анализа является построение
синтаксического дерева предложения. Также как и морфологический анализ,
синтаксический анализ является предварительным этапом перед семантическим
анализом. На этом этапе отсеивается большая часть омонимов (слова разного
значения, но одинаково звучащие, напр., пол, коса, ключ), выявленных на~этапе
морфологического анализа. Что, в~свою очередь, существенно ускорит
семантический анализ.

Для~представления в~памяти компьютера значения всех содержательных единиц
рассматриваемого языка (лексических, морфологических, синтаксических и
словообразовательных) и приведения их к~единому, формальному виду,
понятному компьютеру, используется, специально созданный для~этого
искусственный язык или, как его еще называют некоторые ученые, метаязык.

\section{Семантические категории. Понятие.}
{\sl Семантическая категория}~--- это класс выражений с~однотипными
предметными значениями, при~этом включающий все выражения с~предметным
значением данного типа.

Такими классами являются имена, предикаторы, предметные функторы,
логические термины, повествовательные предложения.

Имена~--- слова и словосочетания, являющиеся знаками предметов.

Предикаторы~--- выражения языка (слова и словосочетания), предметными
значениями которых являются свойства (одноместные предикаторы)
и отношения (многоместные предикаторы).

Свойствами в~современной логике называют характеристики отдельных предметов
(<<белый>>, <<странный>>, <<иметь спинку>>, <<ходить>> и~т.п.).
Отношения~--- это связь между двумя и более предметами (<<находиться
между>>, <<быть братом>>, <<быть больше>>, <<знать лучше, чем>> и~т.п.).
Таким образом, отношения представляют собой характеристики не~отдельных
предметов, а некоторых систем предметов.

Наличие или отсутствие у какого-либо предмета свойства или отношения
к~другим предметам называется признаком. Признаки~--- это любые возможные
характеристики предмета, все, что можно высказать о~предмете.

Предметные функторы~--- это знаки так называемых предметных функций.
Наряду с~математическими функциями сюда относятся такие особые
характеристики предметов, как скорость, плотность, возраст, пол,
профессия, агрегатное состояние, место жительства и др.
Иногда их называют предметно-функциональными характеристиками.

Логические термины (логические константы)~--- это знаки логических
отношений <<и>>, <<или>>, <<если\dots, то\dots>>, <<неверно, что>>
и операций <<всякий>>, <<существует>> (<<некоторые>>), <<тот…, который…>>.

\subsection{Понятия}
\label{subsec:Concept}
\begin{defn}
	\textsl{Понятие}~--- элементарная единица мыслительной деятельности,
    обладающая известной целостностью и устойчивостью и взятая в~отвлечении
    от~словесного выражения этой деятельности. Обычно формирование понятий
    рассматривается как этап рационального познания.
\end{defn}
Понятие имеет объём и содержание. Встречаются понятия с~нулевым объёмом
(вечный двигатель). Выделяют \textit{единичные} и \textit{общие} понятия,
\textit{видовые} и \textit{родовые}, \textit{конкретные} и \textit{абстрактные},
\textit{научные} и \textit{бытовые}. Возможны следующие варианты соотношения
между словом и понятием:
\begin{enumerate}
    \item понятие выражено одним словом;
    \item понятие выражено устойчивым словосочетанием;
    \item понятие выражено громоздким описательным оборотом
    (подготовительная к школе группа в детском саду);
    \item отсутствует словесное выражение понятия, но существуют видовые
    понятия (общее название для фруктовых деревьев во французском);
    \item в~языке отсутствует как понятие, так и его словесное выражение
    (\textit{фр.} falaise).
\end{enumerate}

\section{Синтагматические отношения между понятиями.}
Языковые элементы, следуя один за~другим, образуют определённую
языковую цепочку, последовательность~--- синтагму, внутри которой
составляющие её элементы вступают в~синтагматические отношения.
Они характеризуют связи следующих друг за~другом единиц и определяются
их контрастом; языковой элемент может поэтому противопоставляться либо
предшествующему, либо следующему за~ним, либо и тому и другому одновременно.
Поскольку почти все языковые единицы находятся в~зависимости либо от~того,
что их окружает в~потоке речи, либо от~тех частей, из~которых они состоят
сами, развитие процедур синтагматического анализа идёт по~двум разным линиям:
с~первым свойством связаны методика валентностного анализа и~--- шире~---
свойства сочетаемости языковых единиц, со~вторым~--- понятия и методика
дистрибутивного анализа.

С~понятием синтагматических отношений связаны понятия \textsl{лексической}
и \textsl{семантической} сочетаемости. \textsl{Лексическая сочетаемость}~---
информация о~том, каким должно быть слово~$B$ или класс слов~$B_1 \dots B_n$,
чтобы они могли вступать в~синтаксические отношения со~словом~$A$.
Эта информация задается списком. \textsl{Семантическая сочетаемость}~---
информация о~том, какими семантическими признаками должно обладать
слово~$B$, чтобы вступать в~синтаксические отношения со~словом~$A$.

Сочетаемость лексических единиц определяется их \textsl{валентностью}.
Слово обладает валентностью~$X$, если оно обозначается ситуацию, в~которой
есть обязательный участник в~роли~$X$. Например, в~ситуации покупки
это продавец, покупатель, товар и~т.д. Набор семантических валентностей
слова необходим и достаточен для~описания ситуации. В~тексте
семантическая валентность заполняется семантическим \textsl{актантом}.
Основной список возможных актантов (семантических падежей) описан
Ч.~Филлмором в~работе <<The~case for~case>> (Дело о падеже):
\begin{itemize}
	\item агент (агенс, субъект);
	\item объект;
	\item адресат;
	\item инструмент;
	\item место.
\end{itemize}

Подробно разработана особая система описания синтагматических
свойств единиц на~уровне словоформы и слова в~теории
<<Смысл $\leftrightarrow$ Текст>> (И.А.~Мельчук), где выделено
и определено понятие синтактики. В~частности, по~Ю.Д.~Апресяну,
в~состав синтактики лексемы в~толковом словаре, ориентированном
на~интегральное описание системы языка с~полным согласованием
информации о~словаре и грамматике, должны входить следующие сведения:
\begin{enumerate}
	\item модель управления;
	\item синтаксические признаки;
	\item стандартные лексические функции\nobreakdash-параметры;
	\item сочетаемостные ограничения: коммуникативные, связанные
    с~маркировкой темы и ремы, данного и нового, и прагматические,
    к~которым относятся, в частности, ограничения на~тип речевого
    акта, в~котором может участвовать данная лексема.
\end{enumerate}

В~описании лексической сочетаемости понятие лексических функций
возникло не~только как следствие того, что был обнаружен набор
регулярно присоединяемых стандартных смыслов, но и потому, что
было установлено, что при многих лексемах эти смыслы выражаются
индивидуально (идиоматично), ср.~обозначения совокупностей:
рой пчёл, стая птиц, табун лошадей, стадо коров и~т.п. Сведения
о~реализации лексических функций даны в~<<Толково-комбинаторном
словаре современного русского языка>> И.А.~Мельчука и
A.К.~Жолковского~(1984).

\section{Парадигматические отношения между понятиями}
Парадигматические отношения между словами (и понятиями) проявляются
при~соотнесении их как элементов системы. Поскольку лексическая
парадигматика дана в~тексте в~скрытом виде, для~её выявления
и описания необходимо создание специальных процедур лингвистического
анализа, направленного, в~частности, на~установление значимости
лексических единиц и особенностей их существования как членов
определённых лексико-семантических парадигм. Стоит отметить, что
парадигматическая принадлежность слова влияет на~его синтагматическую
сочетаемость, а из~сочетаемости можно сделать вывод об~отношении
слова к~какой-либо парадигме. В~лексической семантике выделяются
следующие типы парадигматических отношений:
\begin{itemize}
	\item \textsl{Синонимия}~--- тип семантических отношений языковых
    единиц, заключающийся в~полном или частичном совпадении их значений.
    Различаются два основных типа синонимии; семантическая
    (идеографическая) и стилистическая синонимия, выражаемая
    словами с~одинаковой предметной отнесённостью, имеющими
    различную стилистическую характеристику: \textit{верить}~---
    \textit{веровать} (книжн.), \textit{странный}~--- \textit{чудной}~(разг.).
	\item \textsl{Антонимия}~--- тип семантических отношений лексических
    единиц, имеющих противоположные значения (антонимов). Существенные
    различия в~предметах и явлениях объективного мира отражаются
    в~языке как противоположность. Логическую основу образуют
    2~вида противоположности: \textsl{контрарная} и
    \textsl{комплементарная}. Контрарная противоположность
    выражается видовыми понятиями, между которыми есть средний,
    промежуточный член: \textit{молодой»}~--- \textit{нестарый},
    \textit{средних лет}, \textit{пожилой}, \textit{немолодой}\dots~---
    \textit{старый}, ср.~\textit{богатый}~--- \textit{бедный},
    \textit{трудный}~--- \textit{лёгкий}\dots
    Комплементарную противоположность образуют видовые понятия,
    которые дополняют друг друга до~родового и являются предельными
    по~своему характеру. Однако в~отличие от~контрарных понятий
    у~них нет среднего, промежуточного члена: \textit{истинный}~---
    \textit{ложный}, \textit{конечный}~--- \textit{бесконечный},
    \textit{можно}~--- \textit{нельзя} и~т.п.
	\item \textsl{Меронимия}~--- отношение <<часть-целое>>.
    \textsl{Мероним}~--- понятие, которое является составной частью
    другого. \textsl{Холоним}~--- понятие, которое является целым
    над~другим(и) понятием(ями) (то~есть другое(ие) понятие(я)
    является частью первого).
	\item \textsl{Гипонимия}~--- родо-видовое отношение.
    \textsl{Гиперонимом} называют слово с~более широким значением,
    выражающее общее, родовое понятие, название класса (множества)
    предметов (свойств, признаков). \textsl{Гипоним} же~--- понятие,
    выражающее частную сущность по~отношению к другому, более
    общему понятию.
	\item \textsl{Когипонимия}~--- отношение между несколькими
    гипонимами одного гиперонима.
	\item \textsl{Конверсивность}~--- выражается разными
    словами~--- лексическими конверсивами, передающими двусторонние
    субъектно-объектные отношения и обладающими в~силу этого
    как минимум двумя валентностями: <<Наши хоккеисты превосходят
    соперников в скорости>> $\leftrightarrow$ <<Соперники уступают
    нашим хоккеистам в скорости>>.
\end{itemize}

\section{Понятие и слово.}
\textsl{Лексическим значением cлова} называется закрепленная
в~сознании говорящих соотнесенность звукового комплекса языковой
единицы с~тем или иным явлением действительности.
Большинство слов называют предметы, их признаки, количество,
действия, процессы и выступают как полнозначные, самостоятельные
слова, выполняя в~языке номинативную функцию (лат. nominatio~---
называние, наименование). Обладая едиными грамматическими и
синтаксическими значениями и функциями, эти слова объединяются
в~разряды существительных, прилагательных, числительных, глаголов,
наречий, слов категории состояния. Лексическое значение у~них
дополняется грамматическими. Например, слово газета обозначает
определенный предмет; лексическое значение указывает, что это~---
<<периодическое издание в виде больших листов, обычно ежедневное,
посвященное событиям текущей политической и общественной жизни>>.
Существительное газета имеет грамматические значения рода (женский),
числа (этот предмет мыслится как один, а не множество) и падежа.
Слово читать называет действие~--- <<воспринимать написанное, произнося
вслух или воспроизводя про себя>> и характеризует его как реальное,
происходящее в~момент речи, совершаемое говорящим (а не~другими лицами).

Из знаменательных частей речи лишены номинативной функции местоимения
и модальные слова. Первые лишь указывают на~предметы или их признаки:
я, ты, такой, столько; они получают конкретное значение в~речи,
но не~могут служить обобщенным наименованием ряда однотипных предметов,
признаков или количества. Вторые выражают отношение говорящего к~высказываемой
мысли: Наверное, уже пришла почта.

Служебные части речи (предлоги, союзы, частицы) также не~выполняют
номинативной функции, т.е. не~называют предметы, признаки, действия,
а используются как формально-грамматические языковые средства.
Лексические значения слова, их типы, развитие и изменения изучает
лексическая семантика (семасиология) (гр. semasia~--- обозначение +
logos~--- учение).

Грамматические значения слова рассматриваются в~грамматике современного
русского языка.

Все предметы и явления действительности имеют в~языке свои наименования.
Слова указывают на~реальные предметы, на~наше отношение к~ним, возникшее
в~процессе познания окружающего нас мира. Эта связь слова с~явлениями
реальной действительности (денотатами) носит нелингвистический характер,
и тем не~менее является важнейшим фактором в~определении природы слова
как знаковой единицы.

Слова называют не только конкретные предметы, которые можно увидеть,
услышать или осязать в~данный момент, но и понятия об~этих предметах,
возникающие у нас в сознании.

Понятие~--- это отражение в~сознании людей общих и существенных признаков
явлений действительности, представлений об~их свойствах. Такими признаками
могут быть форма предмета, его функция, цвет, размер, сходство или различие
с~другим предметом и~т.д. Понятие является результатом обобщения массы
единичных явлений, в~процессе которого человек отвлекается от~несущественных
признаков, сосредоточиваясь на~главных, основных. Без~такого абстрагирования,
т.е. без~абстрактных представлений, невозможно человеческое мышление.

Понятия формируются и закрепляются в~нашем сознании с~помощью слов.
Связь слов с понятием (сигнификативный фактор) делает слово орудием
человеческого мышления. Без~способности слова называть понятие не~было~бы
и самого языка. Обозначение словами понятий позволяет нам обходиться
сравнительно небольшим количеством языковых знаков. Так, чтобы выделить
из~множества людей одного и назвать любого, мы пользуемся словом человек.
Для~обозначения всего богатства и разнообразия красок живой природы
есть слова красный, желтый, синий, зеленый и~т.д. Перемещение в~пространстве
различных предметов выражается словом идет (человек, поезд, автобус,
ледокол и даже~--- лед, дождь, снег и т.п.

Однако не все слова называют какое-либо понятие. Их~не способны выражать
союзы, частицы, предлоги, междометия, местоимения, собственные имена.
О~последних следует сказать особо.

Есть имена собственные, называющие единичные понятия. Это имена выдающихся
людей (Шекспир, Данте, Лев Толстой, Шаляпин, Рахманинов), географические
названия (Волга, Байкал, Альпы, Америка). По~своей природе они не~могут
быть обобщением и вызывают мысль о~предмете, единственном в~своем роде.

Личные имена людей (Александр, Дмитрий), фамилии (Голубев, Давыдов),
напротив, не~рождают в~нашем сознании определенного представления
о~человеке. Нарицательные~же существительные (историк, инженер, зять)
по~различительным признакам профессий, степени родства позволяют
составить какое-то представление о~людях, названных этими словами.

Клички животных могут приближаться к~обобщенным наименованиям. Так,
если коня зовут Буланый, это указывает на~его пол и масть, Белкой
обычно называют животных, имеющих белую шерсть (хотя так можно назвать
и кошку, и собаку, и козу). Так что разные клички по-разному соотносятся
с~обобщенными наименованиями.

Понятие и слово неотделимы друг от друга в~своем возникновении и
функционировании. Слова являются материальной основой понятий, без~которой
невозможно ни~их образование, ни оперирование ими. Однако единство
понятия и слова не~означает их абсолютного тождества, так как между
ними есть определенные различия. Рассмотрим эти различия более подробно.
Во-первых, не~всякое понятие выражается одним словом. Многие понятия
выражаются словосочетаниями. Например, <<международная организация
уголовной полиции>>, <<комплексный учет всех положений Гражданского
кодекса Российской Федерации>>, «студентка второго курса юридического
факультета Московского гуманитарно-экономического института>> и~др.
Во-вторых, понятие и слово не~всегда однозначно соответствуют друг другу,
что связано с~существованием омонимов и синонимов.

\section{Краткий обзор языков представления знаний}
Под~термином <<представление знаний>> чаще всего подразумеваются способы
представления знаний, ориентированные на~автоматическую обработку
современными компьютерами, и, в~частности, представления, состоящие
из~явных объектов ('класс всех слонов', 'Клайд~--- индивид') и из~суждений
или утверждений о~них ('Клайд~--- слон', 'все слоны серые'). Представление
знаний в~подобной явной форме позволяет компьютерам делать дедуктивные
выводы из~ранее сохранённого знания ('Клайд~--- серый').

\paragraph{Способы представления знаний}
\begin{itemize}
    \item \textsl{Семантическая сеть}~--- каждый узел такой сети
    представляет концепт, а дуги используются для~определения отношений
    между концептами.
    \item \textsl{Фрейм}~--- имеет своё собственное имя и набор
    атрибутов, или слотов которые содержат значения; например фрейм
    {\it дом} мог~бы содержать слоты {\it цвет}, {\it количество этажей}
    и~так далее, слоты могут оставаться незаполненными.
    \item \textsl{Семантические примитивы}~--- слова естественного
    языка, при~помощи которых можно толковать значения всех остальных
    слов, выражений, а также предложений языка.
    \item \textsl{Логика предикатов первого порядка}
\end{itemize}

\paragraph{Специальные языки представления знаний}
\begin{itemize}
    \item Языки описания онтологий
    \begin{itemize}
        \item RDF~--- resource description framework. Ресурсом в~RDF
        может быть любая сущность~--- как информационная (например,
        веб-сайт или изображение), так и неинформационная (например,
        человек, город или некое абстрактное понятие). Утверждение,
        высказываемое о~ресурсе, имеет вид <<субъект~--- предикат~--- объект>>
        и называется \textsl{триплетом}. Утверждение <<небо голубого цвета>>
        в~RDF-терминологии можно представить следующим образом:
        субъект~--- <<небо>>, предикат~--- <<имеет цвет>>, объект~--- <<голубой>>.
        Множество RDF-утверждений образует ориентированный граф,
        в~котором вершинами являются субъекты и объекты, а рёбра
        отображают отношения.
        \item OWL~--- web ontology language~---  язык описания онтологий
        для~семантической паутины. Язык OWL позволяет описывать классы
        и отношения между ними, присущие веб-документам и приложениям.
        В~основе языка~--- представление действительности в~модели
        данных <<объект~--- свойство>>. OWL пригоден для~описания не~только
        веб-страниц, но и любых объектов действительности.
    \end{itemize}
\end{itemize}

\section{Предикаты}
\textsl{Предикат}~--- термин логики и языкознания, обозначающий
конститутивный член суждения~--- то, что высказывается (утверждается
или отрицается) о~субъекте. Предикат находится к~субъекту в~предикативном
отношении, способном принимать отрицание и разные модальные значения.
Понятие предикативного отношения шире, чем понятие предиката, к~которому
предъявляются определённые семантические требования: предикат~--- не~всякая информация
о~субъекте, а указание на~признак предмета, его состояние и отношение
к~другим предметам. Значение существования не~считается предикатом,
а предложения типа <<Пегас (не) существует>>, согласно этой точке зрения,
не~выражают суждения. Не~составляет предиката указание на~имя
предмета (<<Этот мальчик~--- Коля>>) и на~его тождество самому себе
(<<Декарт и есть Картезиус>>). В~ряде современных направлений логики
понятие предиката было заменено понятием пропозициональной функции,
аргументы которой представлены актантами (термами)~--- субъектом и
объектами.

В~языкознании для~некоторых языков (в~западноевропейских терминологических
системах) термин <<предикат>> был использован при~обозначении состава
предложения, соответствующего сообщаемому, а также <<ядерного>> компонента
этого состава. Для~других языков (например, славянских) этот термин был
заменен калькой <<сказуемое>>, что позволило избежать терминологического
смешения логических и грамматических категорий, но не~исключён
из~лингвистического обихода. С~термином сказуемое ассоциируется прежде
всего формальный аспект этого члена предложения, с~термином <<предикат>>~---
его содержательный аспект. Поэтому принято говорить о~формальных типах
сказуемого (ср. глагольное, именное сказуемое), но о~семантических типах
предикатов. Выделяются: таксономические предикаты, указывающие на~вхождение
предмета в~класс (<<Это дерево~--- ель>>); реляционные предикаты,
указывающие на~отношение данного объекта к~другим объектам (<<Пётр~--- отец Насти>>);
характеризующие предикаты, указывающие на~динамические и статические,
постоянные и преходящие признаки объекта (<<Мальчик бежит>>,
<<Мальчик~--- ученик>>, <<Он учит физику>>, <<Он устал>>, <<Ему скучно>>.
В~этом разряде особое место занимают оценочные предикаты: <<Климат
здесь скверный>>); предикаты временной и пространственной локализации
(<<Сейчас полдень>>, <<Павел дома>>). Разные типы предикатов могут быть
представлены в~языке синкретически. Переходные глаголы обычно выражают
не~только определённое отношение между предметами, но также характеристики
этих предметов с~точки зрения данных отношений.

Предикаты могут быть классифицированы и по~другим основаниям. В~зависимости
от~типа субъекта различаются предикаты низшего порядка (относящиеся
к~материальным сущностям) и высшего порядка, характеризующие разные
виды нематериальных объектов, среди которых наиболее резко противопоставлены
предикаты, относящиеся к~событийному субъекту, и предикаты, характеризующие
пропозициональный субъект (ср.: <<Этот случай произошёл вчера>>~---
<<То, что этот случай произошёл вчера, сомнительно>>). По~количеству
актантов предикаты делятся на одноместные (<<Ель — зелена>>),
двухместные (<<Ель заслоняет нору>>), трёхместные (<<Ель заслоняет нору
от~охотника>>) и~т.д. Ю.С.~Степанов разделяет предикаты по~степени
производности в~системе языка на~первопорядковые, т.е. непроизводные
(<<Мальчик учится>>), предикаты второго порядка, т.е. производные
от~первых (<<Мальчик~--- ученик>>), третьего порядка, т.е.
производные от~вторых (<<Это~--- лишь ученичество>>) и~т.д.
\section{Статья Мочаловой.}
В~работе предлагается алгоритм поиска семантических зависимостей
между частями предложений анализируемого текста, основанный
на~сопоставлении текста с~базовыми семантическими шаблонами.
Каждое предложение, поступающее на~вход программе в~ходе анализа,
постепенно сокращается: некоторые части предложения в~соответствии
с~правилами, описанными в~семантических шаблонах, добавляются
в~очередь с~приоритетом, после чего на~каждой итерации алгоритма
из~анализируемого предложения изымается та его часть, которая
имеет в~очереди наибольший приоритет. Для~определения приоритета
в~такой очереди используются два значения: значение приоритета
группы, к~которой принадлежит семантическая зависимость, описанная
в~шаблоне и позиция слова (или последнего слова из~набора, если
элемент, хранимый в~очереди, состоит из~нескольких слов)
в~анализируемом предложении.

Базовым семантическим шаблоном назовем правило, по~которому
в~анализируемом тексте находится семантическая зависимость.
Он состоит из~4~основных частей:
\begin{enumerate}
    \item последовательность слов или неделимых смысловых единиц,
    для~которых указаны их морфологические признаки, а в~некоторых
    случаях приведены названия этих слов и смысловых единиц;
    \item название семантического отношения;
    \item последовательность чисел, определяющая позиции
    в~последовательности из~п.1, элементы которой должны быть
    добавлены в очередь с~приоритетом, в~соответствии с~которой
    впоследствии будут удаляться слова из~анализируемого предложения,
    подаваемого на~вход семантическому анализатору;
    \item число, обозначающее значение приоритета, группы
    семантических зависимостей, к~которой относится данное
    семантическое отношение.
\end{enumerate}

<<очередь с приоритетом>> используется для~хранения слов или
набора слов, являющихся правым аргументом некоторой семантической
связи, найденной в~анализируемом предложении. Для~определения
приоритета элемента в~такой очереди используются два значения:
\begin{itemize}
    \item значение приоритета группы, к~которой принадлежит
    данная семантическая связь;
    \item позиция слова (или последнего слова из~набора,
    если элемент, хранимый в~очереди, состоит из~нескольких слов)
    в~анализируемом предложении.
\end{itemize}

Будем говорить, что элемент из~описываемой очереди обладает
наивысшим приоритетом, если значение приоритета семантической
группы минимально, а значение позиции последнего слова из~набора,
образующего элемент, максимально. Т.о., элементы очереди с~приоритетом
сортируются по~возрастанию приоритетов групп семантических зависимостей.
В~случае если в~очереди нашлось несколько элементов с~одинаковыми
значениями приоритетов семантических групп, то тогда такие элементы
сортируются по~убыванию позиции последнего слова, относящегося
к~рассматриваемому элементу, в~анализируемом предложении.

\subsection{Алгоритм нахождения семантических зависимостей с помощью базовых семантических шаблонов}
\paragraph{Обозначения:}
\begin{enumerate}
    \item $T$~--- анализируемое предложение;
    \item $t_i$~--- $i$-ая неделимая смысловая единица
    анализируемого предложения $T$;
    \item $S$~--- множество всех базовых семантических
    шаблонов;
    \item $s_i$~--- $i$-ый шаблон множества $S$;
    \item $sp_i$~--- приоритет шаблона $s_i$;
    \item $R_i(t_{i_1},t_{i_2})$~--- семантическая зависимость~$R_i$,
    определяемая шаблоном~$s_i$ и связывающая две неделимые
    смысловые единицы~$t_1$ и~$t_2$;
    \item $pos(t_i)$~--- позиция в анализируемом предложении последнего слова из $t_i$;
    \item $Q$~--- очередь с приоритетами;
    \item $(t_{i_2}, sp_i, pos_i)$~--- элемент очереди $Q$, образованный
    посредством шаблона $s_i$;
\end{enumerate}
\paragraph{Алгоритм нахождения семантических зависимостей:}
На~вход семантическому анализатору подается предложение~$T$
на~естественном языке и множество $S={s_1,s_2,\dots}$ базовых
семантических шаблонов, где~$s_i$~--- отдельный семантический
шаблон. Далее предложение~$T$ разделяется на~неделимые
смысловые единицы, обозначаемые~$t_i\;i=1\dots n$ , состоящие
либо из~одного слова, либо из~набора слов, который может
являться именованной сущностью (например, название государства,
название мероприятия, титул человека и~т.п.).

После формирования набора неделимых смысловых единиц производится
морфологический анализ каждой такой единицы, после чего в~модуле
выделения языковых конструкций происходит поиск таких сложных
языковых конструкций, как вводные, причастные и деепричастные
обороты, придаточные предложения и~т.д.

Далее последовательно происходит поиск совпадений каждого базового
семантического шаблона из множества~$S$ в множестве~$T$, при~этом
как для~базовых семантических шаблонов, так и для всех~$t_i$
учитываются морфологические характеристики.

В~случае обнаружения совпадения семантического шаблона~$s_i$,
с~некоторыми подмножествами~$t_i$ множества~$T$ формируются
семантические зависимости $R_i(t_{i_1}, t_{i_2})$, которые
записываются в~БД, если они обнаружена в~анализируемом тексте
впервые.

При этом в очередь~$Q$ с приоритетом  добавляется новый элемент,
представленный тройкой $(t_{i_2}, sp_i, pos(t_{i_2}))$, где $sp_i$~---
приоритет группы, к которой относится семантическая зависимость $R_i$.
Поиск базовых семантических шаблонов в~$T$ происходит до~тех пор,
пока не~будут проверены на~совпадение все шаблоны. После окончания
поиска в~$T$ шаблонов происходит проверка очереди~$Q$ с~приоритетом
на~пустоту с~помощью функции isEmpty: если оно пусто, то это
означает, что на~очередном этапе сопоставления шаблонов с~$T$
новых семантических зависимостей не~найдено, и программа завершает
свою работу. В противном случае посредством функции remove
получаем элемент $(del,sp_{del}, pos_{del})$ из~очереди $Q$
с~наивысшим приоритетом, после чего значение~$del$ удаляется
из~текущего множества~$T$, представляющего оставшиеся
для~дальнейшего анализа слова из анализируемого предложения.

Далее повторяется поиск базовых семантических шаблонов~$S$
среди оставшихся неделимых семантических единиц множества $T$.
Так~продолжается до~тех пор, пока множество~$Q$ не~станет
пустым (это означает, что в анализируемом предложении найдены
все возможные семантические зависимости, описанные базовыми
семантическими шаблонами~$S$).

\section{Статья Шерстиновой.}
\textsl{Речевой корпус <<Один речевой день>> (ORD корпус)}
разрабатывается с~целью исследования повседневной устной речи
и бытовой коммуникации. Методологической основой создаваемого
корпуса является осуществление звукозаписей повседневной речи
в~условиях, максимально приближенных к~естественным, для~чего
используется методика непрерывной 24\nobreakdash-часовой записи
всей речевой коммуникации информантов в~течение суток.
К~настоящему времени записано более 300~часов звучания,
полученных от~40~информантов (20~мужчин и 20~женщин). Звукозаписи
переформатированы, убраны длительные (больше 5~минут) шумовые
фрагменты, не~содержащие речи. Звукозаписи разрезаны
на~коммуникативные эпизоды по~принципу общих условий коммуникации
и качества звукозаписи. В~результате было получено более
900~файлов-эпизодов.

Для~аннотирования корпуса ORD используются два профессиональных
программных продукта:
\begin{itemize}
    \item программа многоуровневого лингвистического
    аннотирования \textbf{ELAN},
    \item программа профессионального фонетического
    анализа \textbf{Praat}.
\end{itemize}

Первичное аннотирование (расшифровка) данных осуществляется
в~программе ELAN и предполагает заполнение следующих уровней:
\begin{itemize}
    \item Frase~--- отделение реплик говорящих от~неречевого сигнала,
    \item Speaker~--- кодирование говорящего,
    \item Voice~--- определение качества голоса,
    \item Events~--- разметка неречевых аудиособытий,
    \item FonetComment~--- отклонения от литературной нормы,
    \item FraseComment~--- информация о реализации конкретной реплики,
    \item Notes~--- общий комментарий,
    \item Episode~--- обозначение мелких эпизодов и
    мини\nobreakdash-сценариев.
\end{itemize}


\section{Статья Мазова.}
В~настоящее время многочисленные информационные органы
и библиотеки как в~России, так и за~рубежом используют
различные СУБД, основу которых составляют файлы в~структуре
стандарта ISO-2709. Широкое использование формата ISO-2709
в~библиотечных системах обусловлено тем, что библиографическая
информация является свободнотекстовой и слабоструктурированной,
что не~позволяет эффективно использовать для~ее обработки
реляционные СУБД. Стандарт ISO-2709 лежит в~основе всех форматов
для~библиографических записей семейства MARC (MAchine Readable
Cataloguing), таких как USMARC, UNIMARC, RUSMARC и~др. Однако
формат ISO-2709 имеет ряд существенных ограничений (например,
на~длину записи и уровень иерархии) и является сложночитаемым
для~пользователя.

В~настоящее время наиболее универсальным средством кодирования
и отображения содержания информационных документов является язык~XML.
Иерархическая структура библиографической записи хорошо согласуется
с~моделью XML-документа.

Использование XML в~качестве формата обмена и хранения библиографических
данных позволяет осуществлять контроль корректности записей
на~уровне проверки XML-документа. В~отличие от~формата ISO-2709,
XML~--- это формат, читаемый для~человека и легко документируемый.

В~отличие от~большого разнообразия используемых MARC-форматов, XML
стандартизирован и поддерживается большим количеством производителей
программного обеспечения. В~стандарт XML включена поддержка Unicode,
что позволяет создавать многоязычные документы, а также использовать
расширенный набор символов.

При~использовании конвертеров, свободно распространяемых в~сети
Интернет, чтобы получить запись в~нужном представлении в~формате XML,
необходимо воспользоваться как минимум двумя конвертерами, один из~которых
изменяет формат данных (между ISO-2709 и XML), а второй меняет внешнее
представление данных. Основной отличительной чертой этого программного
приложения является возможность осуществлять эти преобразования одновременно.

Кроме того, приложение предусматривает различные кодировки данных,
в~частности кириллицу (отсутствие данной возможности не~позволяет
эффективно эксплуатировать разработки зарубежных авторов, представленные
в~сети Интернет), и имеет достаточно подробную справочную систему.


\end{document}
