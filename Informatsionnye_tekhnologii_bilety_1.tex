% ----------------------------------------------------------------
% Article Class (This is a LaTeX2e document)  ********************
% ----------------------------------------------------------------
\documentclass[12pt]{article}
\usepackage[english, russian]{babel}
\usepackage{amsmath,amsthm}
\usepackage{amsfonts}
\usepackage{indentfirst}
\usepackage{enumitem}
\usepackage[colorlinks,urlcolor=blue]{hyperref}
\hypersetup{
    colorlinks,
    citecolor=black,
    filecolor=black,
    linkcolor=black,
    urlcolor=black
}
\usepackage[utf8]{inputenc}

% THEOREMS -------------------------------------------------------
\newtheorem{thm}{Theorem}[section]
\newtheorem{cor}[thm]{Corollary}
\newtheorem{lem}[thm]{Lemma}
\newtheorem{prop}[thm]{Proposition}
\theoremstyle{definition}
\newtheorem*{defn}{Определение}
\theoremstyle{remark}
\newtheorem{rem}[thm]{Remark}
\numberwithin{equation}{section}
% ----------------------------------------------------------------
\begin{document}

\title{Информационные технологии в лингвистике}%
\author{Билеты к экзамену}%
\date{}
% ----------------------------------------------------------------
\maketitle
% ----------------------------------------------------------------
\tableofcontents

\section{Лингвистические информационные технологии; лингвистические ресурсы в Интернете}
\subsection{Информационные технологии в лингвистике}
В~70-е годы XX века началось бурное развитие компьютерных технологий.
Компьютеры позволили обрабатывать огромные массивы информации, прежде
неподвластные обработке. Это дало скачок в~развитии некоторых областей
прикладной лингвистики, ранее находившихся в~зачаточном состоянии, и
позволило по-новому взглянуть на~ряд фундаментальных задач.
\begin{defn}
    {\sl Информационные технологии в лингвистике}~--- это совокупность
    законов, методов и средств получения, хранения, передачи,
    распространения, преобразования информации о~языке и законов
    его функционирования с~помощью компьютеров.
\end{defn}

Информационные технологии находят свое применение прежде всего
в~прикладных задачах. К~их числу можно отнести:
\begin{itemize}
    \item создание систем искусственного интеллекта;
    \item создание систем автоматического перевода;
    \item создание систем автоматического аннотирования и реферирования
    текстов;
    \item создание систем порождения текстов;
    \item создание систем обучения языку;
    \item создание систем понимания устной речи;
    \item создание систем генерации речи;
    \item создание автоматизированных информационно-поисковых систем;
    \item создание систем атрибуциии дешифровки анонимных и
    псевдоанонимных текстов;
    \item разработка различных баз данных (словарей, карточек, каталогов
    реестров, и~т.п.) для гуманитарных наук;
    \item разработка различного типа автоматических словарей;
    \item разработка систем передачи информации в~сети Интернет и~т.п.
\end{itemize}

Самое слово {\sl информация} означает некоторые сведения о~внешнем и
внутреннем мире, которые мы используем для~регулирования своего
поведения. Она определяется такими параметрами: ценность, достоверность,
полнота, актуальность, логичность, компактность.

С~современной точки зрения {\sl информатика}~--- наука о~законах и
методах получения, хранения, передачи, распространения и использования
информации в~естественных и искусственных системах с~применением
компьютера.

\subsection{Лингвистические ресурсы в интернете}
Вот некоторые лингвистические ресурсы в интернете:
\begin{enumerate}[label=\*]
    \item \href{http://linguistlist.org/}{The Linguist List}~--- большой
    каталог программного обеспечения для~различных областей компьютерной
    обработки текстов и лингвистики.
    \item \url{http://www.sil.org/computing/}~--- обширный каталог
    программ по~вычислительной лингвистике, разработанных в~рамках
    проекта SIL.
    \item \url{http://www.sil.org/linguistics/computing.html}~--- очень
    большая коллекция ссылок на~программы чрезвычайно широкой лингвистической
    направленности в~сети Internet.
    \item \url{http://www.content-analysis.de/index.html}~--- электронный
    каталог ресурсов, связанных с~анализом текстов (на~англ.языке).
    \item \href{http://www.gramota.ru/index.html}{Справочно-информационный
    портал <<Русский язык>>}~--- замечательный ресурс для~истинных
    любителей русской словесности, содержащий массу полезной информации.
    Также включает on-line словари русского языка.
    \item \href{http://www.textanalysis.info/}{Text Analysis Info}~---
    бесплатный информационный портал, посвященный анализу контента
    межчеловеческого общения. Также представлены различные программы,
    предназначенные для обработки таких источников как аудио-, видео- или
    речевых данных.
    \item \href{http://www.lti.cs.cmu.edu/research/projects}{LTI Projects}~---
    каталог проектов, посвященных созданию систем машинного перевода,
    обработки речи, информационного поиска, извлечения знаний и других.
    \item \url{http://ruscorpora.ru/}~--- национальный корпус русского языка.
    \item \url{http://opencorpora.org/}~--- открытый корпус русского языка.
\end{enumerate}

\section{Типология программных средств для лингвистических информационных технологий}
Существует немало программных средств, предназначенных для~обеспечения
анализа, обработки, хранения и поиска текстов, рисунков и аудиоданных
на~естественном языке.
\paragraph{Программы анализа и лингвистической обработки текстов.}
Наиболее важные парсеры этого типа производят морфологическую разметку
текста, по~возможности снимая неоднозначность на~разных уровнях, или
строят графы зависимостей, выделяют имена собственные и решают более
узконаправленные задачи.

\paragraph{Программы преобразования текстов.}
Решают задачи поиска и замены в~тексте определенных элементов,
смены кодировки, обработку разметки и т.п.
\paragraph{Проверка орфографии.}
Ряд парсеров проверяет пунктуацию, орфографию, грамматику и оценивает
стилистику текста.
\paragraph{Автоматические генераторы текстов}
Еще некоторое время назад многие подобные программы имели шуточный
характер: они генерировали стихи или <<философские>> изречения,
но сейчас и это направление бурно развивается. Крупные компании
используют автоматические генераторы канцелярских документов, отчетов
и даже автореферирование и рецензирование.
\paragraph{Машинный перевод.}
Программы и сервисы, предназначенные для~автоматического перевода
текстов с~одного естественного языка на~другой.
\paragraph{Словари и тезаурусы}
Электронные словари, теперь бывают и статистические.
\paragraph{Поисковые машины и системы полнотекстового поиска.}
Поиск по~ключевой информации, кластеризация результатов.
\paragraph{Синтез и распознавание речи.}
Направление исследований: распознавание $\rightarrow$ перевод
$\rightarrow$ синтез.
\paragraph{Распознавание символов}

\section{Автоматический анализ текста; уровни анализа; взаимодействие
между уровнями}
Отражение уровневой организации языка в~архитектуре систем
компьютерного анализа естественно-языкового текста:
\begin{itemize}
    \item фонетический уровень;
    \item графематический уровень;
    \item морфологический уровень;
    \item синтаксический уровень;
    \item семантический уровень.
\end{itemize}

\paragraph{Фонетика}~--- раздел языкознания, изучающий речевые
звуки и звуковое строение языка (слоги, звукосочетания, закономерности
соединения звуков в речевую цепочку). Различает звуки гласные
и согласные, звонкие и глухие и т.д. Изучает:
\begin{itemize}
    \item звук речи с~точки зрения его создания, какие органы речи
    участвуют в~его произношении;
    \item звук как колебание воздуха и фиксирует его физические
    характеристики: частоту (высоту), силу (амплитуду), длительность;
    \item функции звуков в~языке, оперирует фонемами, т.е.
    минимальными языковыми единицами, обладающими смыслоразличительной
    функцией.
\end{itemize}
\paragraph{Графематический анализ.} Этап графематического
анализа предназначен для~выделения элементов структуры текста:
параграфов, абзацев, предложений, отдельных слов и~т.д.
В~задачу графематического анализа входят:
\begin{itemize}
    \item Выделение абзацев, заголовков, примечаний;
    \item Выделение предложений из~входного текста;
    \item Разделение входного текста на~слова, цифровые комплексы,
    формулы и~т.д. {\sl Токенизация};
    \item Сборка слов, написанных в~разрядку;
    \item Выделение устойчивых оборотов, не~имеющих словоизменительных
    вариантов;
    \item Выделение ФИО (фамилия, имя, отчество), когда имя и отчество
    написаны инициалами;
    \item Выделение иностранных лексем, записанных латиницей;
    \item Выделение электронных адресов и имен файлов;
\end{itemize}

\paragraph{Морфологический анализ.} Единица~--- слово. На~этом этапе
обрабатываются отдельные слова, в~них выделяются основы и флексии
(изменяемые части слов)~--- приставки, суффиксы, окончания.
В~дальнейшем флексии используются для~установления грамматических
отношений между словами в~рамках одного предложения.

Слова как единицы грамматические и лексические группируются
в~части речи, т.е. в грамматические классы слов, объединяющиеся
на~основании обобщенных значений. Обобщенное значение, характеризующее
все слова той или иной части речи, есть абстрактное представление
того общего, что присутствует в~лексических и морфологических значениях
конкретных слов данного класса. Наиболее обобщенными значениями
для~частей речи являются значения предмета (субстанции) и признака~---
процессуального (представляемого как действие или состояние) и
непроцессуального (представляемого как качество или свойство).

Морфологический анализ обеспечивает определение нормальной
формы, от~которой была образована данная словоформа, и набора
параметров, приписанных данной словоформе. Нормальная форма
(именительный падеж для~существительных, инфинитив для~глаголов
и~т.д.) называется {\sl леммой}, а сам процесс определения лемм~---
{\sl лемматизацией}.

\paragraph{Синтаксический анализ.} Единица~--- предложение.
В~результате синтаксического анализа линейная последовательность
токенов (слов) преобразуется в~набор синтаксических отношений.
Грамматично построенные предложения являются связными, т.е.
лишенными разрывов в~цепочке синтаксических отношений. Отношения
являются бинарными. Синтаксическое отношение неравноправно:
определяемое слово <<главнее>> своего определения. Важная
особенность синтаксических зависимостей заключается в~том,
что они далеко не~всегда связывают слова, находящиеся рядом
в~цепочке.

Выполнение задачи осложняется огромным количеством альтернативных
вариантов, возникающих в~ходе разбора, связанных как
с~многозначностью входных данных (одна и та же словоформа может
быть получена от~различных нормальных форм), так и неоднозначностью
самих правил разбора.

\paragraph{Семантический анализ.} Занимается решением задач,
связанных с~возможностью определения значения слова в~зависимости
от~контекста и конкретной ситуации, понимания смысла фразы.
Элемент значения языкового знака называется {\sl семой}.

Во многих случаях смысловой элемент состоит из нескольких слов.
Последовательность из двух или более слов, частотность совместного
появления которых в тексте выше, чем ожидаемая вероятность их
совместного появления, называется {\sl коллокацией}. В отличие
от свободного словосочетания, коллокация определяет, какие слова
могут быть использованы вместе, например, с какими предлогами
управляет тот, или иной глагол, или какие глаголы и существительные
обычно используются вместе.

С помощью компьютерных технологий коллокации могут автоматически
извлекаться из текстов. Для этого используются различные меры
ассоциативной связи, которые оценивают, является ли взаимное появление
лексических единиц случайным, или оно статистически значимо. Однако
часто статистически значимое совместное появление двух слов не образует
коллокации.

\section{Фонетический уровень анализа языка и речи}

Фонетический уровень описывает звуковую сторону языка. {\sl Фонетика}~---
наука о~<<звуковом материале>> языка, об~использовании этого
материала в~значащих единицах языка и речи, об~исторических изменениях
в~этом материале и в~приемах его использования. Звуки и другие звуковые
единицы (например, слоги), а также такие звуковые явления, как ударение
и интонация, изучаются фонетикой с~нескольких разных точек зрения:
\begin{itemize}
    \item {\sl акустическая фонетика}~--- изучает звуки как физические
    явления;
    \item с~точки зрения работы, производимой человеком при~их произнесении
    и слуховом восприятии, т.е. в~биологическом аспекте~--- {\sl артикуляторная
    фонетика};
    \item и самое главное~--- с~точки зрения их использования в~языке, т.е.
    отношений между собой в~речевой цепи~--- {\sl фонология}.
\end{itemize}

Фонетика имеет большое практическое значение. Без~нее были~бы
невозможны правильная методика обучения письму и чтению, постановка
произношения при~изучении неродного языка, создание рациональной
системы письма для~бесписьменных языков и усовершенствование
существующих систем письма, успешное лечение дефектов речи и~т.д.
При~изложении различных вопросов фонетики приходится во~многих
случаях пользоваться специальными видами письма~--- той или иной
научной транскрипцией. Это связано с~тем, что в~обычном письме
между буквой и звуком часто нет однозначного соответствия.
Например, в~русском письме один и тот~же звук нередко записывается
разными буквами (скажем, буквой {\bf в} в слове {\bf плечевой} и буквой {\bf г}
в слове {\bf чего}), и наоборот, одна и та~же буква читается как разные
звуки (скажем, буква {\bf г} в~словах {\bf игра}, {\bf чего},
{\bf легкий}, {\bf снег}).

\subsection{Методика записи устной речи}
Имеется два направления исследования речи:
\begin{itemize}
    \item Исследовать записанную дикторами речь, то есть произношение
    текста отдельными информантами,
    \item Исследовать спонтанную речь, записывая живые разговоры.
\end{itemize}
В этом пункте мы рассмотрим первый тип.

\subsection{Этапы исследования устной речи}
Создание речевых баз данных (или, иначе, речевых корпусов) представляет
собой определенный технологический процесс. В~нем можно выделить следующие
основные этапы:
\begin{itemize}
    \item подготовка фонетического обеспечения для~формирования речевого
    корпуса, транскрипция;
    \item подготовка текстового материала;
    \item разработка программного обеспечения для~формирования речевого
    корпуса;
    \item подбор дикторского состава;
    \item запись речевых фрагментов, произнесенных дикторами;
    \item проверка качества записи речевых фрагментов;
    \item фонетическая верификация речевых фрагментов и их разметка;
    \item обработка результатов верификации;
    \item окончательное формирование речевого корпуса.
\end{itemize}
\subsection{Основные проблемы}
На каждом этапе возникают определенные проблемы.
\begin{itemize}
    \item Формирование стандартов записи звуков;
    \item в~зависимости от~целей исследования необходимо подобрать
    наиболее оптимальный текст: труднопроизносимые слова и слишком длинные
    предложения могут быть записаны с~ошибками;
    \item запись надо очистить от~шумов и проверить на~оговорки;
    \item аннотирование и~т.п.
\end{itemize}

\subsection{Отбор информантов}
В зависимости от~задач подбираются те или иные информанты. Например,
для~исследования определенного акцента, произношения или дефекта,
требуются одни носители языка, а для исследования норм или для~распознавания
речи~--- иные.

\subsection{Фонетические базы данных}
Имеется несколько речевых баз данных для~русского языка, среди которых
ISABASE, МУРКО, УМКО, РЭК, <<Один речевой день>> и другие.

\subsection{Способы преобразования звучащей речи в~текст}
В~зависимости от~объемов и задач расшифоровка бывает автоматическая,
полуавтоматическая и полностью ручная. Среди автоматических парсеров
используют Praat и ELAN. Этапы расшифровки:
\begin{itemize}
    \item frase (реплики и время начало-конца)
    \item speaker (кодирование говорящего)
    \item voice (качество голоса)
    \item events (неречевые аудио-события)
\end{itemize}

\subsection{Структура и организация речевых баз данных}
Большинство крупномасштабных речевых корпусов формируется с~целью их
использования для~решения задач распознавания речи, и поэтому они обычно
содержат разделы, предназначенные для~обучения систем распознавания и
для~последующего тестирования качества работы этих систем.

Каждый из~этих двух разделов обычно еще разбивается на~подразделы, каждый
из~которых, в~свою очередь, содержит речевые фрагменты, произнесенные
одним диктором. При~этом желательно поддерживать выполнение двух требований.
Во-первых, не должны пересекаться составы дикторов, относящихся к~этим
разделам, а, во-вторых, наборы предложений, произнесенных дикторами,
относящимися к~разделу обучения и к~разделу тестирования, должны быть
различны.

Иногда бывает удобно разделы обучения и тестирования структурировать
дополнительно, разбивая их на~подразделы, соответствующие, скажем,
дикторам-мужчинам и дикторам-женщинам, или на~подразделы, соответствующие
различным диалектам дикторов.

Бывает целесообразно также, наряду с~основными разделами обучения и
тестирования формировать еще один специальный раздел речевой базы,
предназначенный для~целей отладки и совершенствования самой системы
распознавания речи. В~этом случае желательно, чтобы состав дикторов,
соответствующих этому разделу, не~пересекался с~составами дикторов
для~разделов обучения и тестирования. Это ограничение накладывается
также и на~множество предложений, подготовленных для~такого раздела.

Если формируемая речевая база предназначена не~только для~использования
в~системах автоматического распознавания речи, но~и для~теоретических
исследований в~области фонетики, то желательно структурировать ее еще
и в~соответствии с~фонетическими особенностями лексического материала,
такими, например, как фонетическая полнота, фонетическая сбалансированность,
фонетическая репрезентативность и~т.п. В~этом случае в~те или иные разделы
базы заносятся речевые фрагменты, образованные дикторами при~произнесении
наборов предложений, обладающих соответствующими фонетическими особенностями.

\subsection{Применение звуковых баз данных}
Позволяет тестировать и разрабатывать методы распознавания речи,
изучать лингвистическую динамику записанного материала:
исследовать временные ряды количественных переменных с~помощью
стандартных статистических методов и анализировать частотные ряды
(лексики, грамматических и, в~частности, синтаксических структур,
семантики или разговорных тем, тех или иных акустических явлений или
просодических контуров) в~зависимости от~времени суток и условий
коммуникации в~самом широком понимании этого термина, а также решает
множество других задач, таких как анализ влияния профессии на~бытовую
жизнь человека, получение информации о среднем артикуляционном темпе
спонтанной речи носителей русского языка.

\section{Корпусы звучащей речи для русского языка; корпус <<Один речевой день>>,
аннотирование корпуса, программное обеспечение}
В~этом разделе мы продолжим обсуждать речевые базы данных. Корпус
<<Один речевой день>> исследует спонтанную человеческую речь.
Производится отбор анонимных информантов, незнакомых группе
исследователей. Им крепится диктофон и далее, в~течение дня,
производится запись всех разговоров. Затем исследователь производит
расшифровку записи и описывает сцены.

Аннотирование производится так. Во-первых, заполняется ряд таблиц
об~информаторе и участвующих в~записи лиц. Каждой реплике
задается тайм-код, ее расшифровка и некоторые специальные
символы: обозначение пауз, вопросов и восклицаний, растягивание
гласных и~т.п.

Хранение информации производится в~MS~Access.

\section{моделирование в лингвистике;  свойства лингвистических
моделей}
Современная структурная лингвистика~--- по~большей части, наука
о~моделях языка. Модель~--- формализованное описание объекта,
системы объектов, выраженное конечным набором предложений какого-либо
языка, формулами, таблицами, графиками и~т.д.

\paragraph{Свойства лингвистических моделей.}
\begin{enumerate}
	\item Моделировать можно только такие явления, существенные
    свойства которых исчерпываются их структурными характеристиками
    и никак не~связаны с~их физической природой. Моделью такого объекта
    должно считаться любое устройство, функционально похожее на~него
    (функциональная аппроксимация объекта).
	\item Модель всегда является некоторой идеализацией объекта
    (например, принцип подстановки недостающих членов предложения
    в~синтаксисе; предположение о~бесконечном числе возможных правильных
    предложений языка).
	\item Модель оперирует не~понятиями о~реальных объектах, а конструктами,
    понятиями об~идеальных объектах (например, конструктом является
    понятие нулевой флексии).
	\item Всякая модель должна быть формальной~--- явно и однозначно
    заданы исходные объекты, связывающие их утверждения и правила
    обращения с~ними.
	\item Модель должна обладать объяснительной силой, то есть объясняет
    факты, необъяснимые старой теорией, и предсказывает неизвестное раньше,
    но возможное поведение объекта.
\end{enumerate}

\paragraph{Построение модели.}
Основные шаги: фиксирование фактов, требующих объяснения; выдвижение гипотез
для~объяснения фактов; реалиация гипотез в~виде моделей; экспериментальная
проверка модели.

Типы лингвистических моделей:
\begin{itemize}
	\item модели, имитирующие речевую деятельность человека (моделируем
    конкретные языковые процессы и явления);
	\item модели исследования (моделируем исследовательскую деятельность
    лингвиста);
	\item метатеории.
\end{itemize}

Кроме того, выделяются аналитические, синтетические и порождающие модели.

\section{Представление письменного текста в компьютерных системах;
кодировки операционных систем; UNICODE}
\begin{defn}
\textsl{Письмо}~--- знаковая система фиксации речи, позволяющая с~помощью
начертательных (графических) элементов передавать речевую информацию
на~расстоянии и закреплять её во~времени. Существует 4 основных типа
письма~--- идеографический, словесно-слоговой, собственно силлабический
и буквенно-звуковой (алфавитный).

{\sl Набор символов} (англ. {\sl character set})~--- таблица, задающая
кодировку конечного множества символов элементов текста: букв, цифр,
знаков препинания и др.

{\sl Кодовая страница}~--- набор одновременно применяемых цифровых символов,
каждому из~которых соответствует цифровой код.
\end{defn}

Кодировкой называют как стандарт кодирования, так и соответствующий набор
символов.

\subsection{Кодировки}
До~середины девяностых годов в~основном использовался стандарт \textbf{ASCII}
(American Standard Code for Information Interchange), который нормирует систему
кодирования букв английского алфавита, цифр от 0 до 9, знаков препинания, а также
служебных (управляющих) символов. Для языков с~латинским алфавитом была создана
расширенная таблица, в~которой некоторые служебные знаки заменены на~буквы
национальных алфавитов, чаще всего буквы с~диакритическими знаками. Для~языков,
в~которых используются другие алфавиты, применялись кодовые страницы, в~которых
половину кодовой таблицы (коды 0--127) занимают буквы латинского алфавита,
а другую половину (128-255)~--- дополнительные символы, в~частности буквы любого
другого алфавита. Таким образом, кодовая страница содержала максимум 256 символов,
включая управляющие символы. Такое число кодов позволяло представить два алфавита,
например, английский и русский, однако при~этом представление других знаков
было невозможным. В языках разметки (html, xml) в~настоящее время для записи
специальных символов и букв с~диакритическими знаками применяются идентификаторы,
например, \texttt{agrave}, \texttt{aacute}.

В~1991~г. некоммерческой организацией Unicode Consortium, Unicode предложен стандарт
\textbf{Unicode} (Юникод), который содержит коды для разных систем письма.
В~первой версии Unicode, которую называют \textbf{UTF\nobreakdash-8} (Unicode
Transformation Format) принята кодировка с~фиксированным размером символа в~16 бит,
т.е. общее число кодов было $2^{16}$ (65 536 символов). Существует практика
обозначения кодов символов числами в~шестнадцатеричной системе (например, U+04F0).
Во~второй версии UTF\nobreakdash-16 (1996 г.) кодовая область значительно расширена.
В~UTF\nobreakdash-16 можно отобразить 1 112 064 символов, что практически полностью
охватывает все современные и исторические системы письма, а также математическую
и музыкальную нотации. В настоящее время Unicode является наиболее распространенным
стандартом, но и стандарт ASCII не вышел из употребления.

\subsection{UNICODE}
Стандарт состоит из~двух основных разделов: \textsl{универсальный набор символов}
(англ. UCS, universal character set) и \textsl{семейство кодировок} (англ. UTF,
Unicode transformation format). Универсальный набор символов задаёт однозначное
соответствие символов кодам~--- элементам кодового пространства, представляющим
неотрицательные целые числа. Семейство кодировок определяет машинное представление
последовательности кодов UCS.

Коды в~стандарте Юникод разделены на~несколько областей. Область с кодами от
U+0000 до U+007F содержит символы набора ASCII с соответствующими кодами.
Далее расположены области знаков различных письменностей, знаки пунктуации и
технические символы. Часть кодов зарезервирована для~использования в~будущем.

\section{Транслитерация и транскрипция; проблема различного написания слов;
способы ее решения}
\begin{defn}
\textsl{Транслитерация}~--- конверсия систем письма, при~которой каждый
графический элемент (знак) одной системы письма представляется (заменяется)
одним и тем же графическим элементом другой системы письма.

\textsl{Ослабленная транслитерация}~--- замена некоторых букв исходного текста
сочетанием двух или более букв чужого алфавита.

\textsl{Строгая транслитерация}~--- замена каждой буквы исходного текста
только одной буквой другой письменности.

\textsl{Транскрипция}~--- способ однозначной фиксации на~письме звуковых
отрезков речи.
\end{defn}

\subsection{Проблема различного написания слов и способы ее решения.}
Чаще всего разночтения встречаются в~написании собственных имен. Некоторые
из~них (изменения имен, псевдонимы, переименования географических объектов и~т.д.)
мало имеют отношения собственно к~компьютерной лингвистике, хотя чрезвычайно
существенны при создании информационно-поисковых языков.

Для~унификации транслитерации создан стандарт \textbf{ISO~9:95}. В~нем
описаны 2 системы транслитерации: система А для строгой транслитерации
и система В для ослабленной.

Для~представления иностранных имен используется не~только транслитерация,
но и транскрипция, т.е. фиксация на~письме звучания слова. В~реальности
эти два подхода сосуществуют, а порой смешиваются.

Передача на~русском языке имен Ивлин Во (Evelyn Waugh) или Шарль Бодлер
(Charles Baudelaire) является примером транскрипции. С~другой стороны,
при~передаче на~русском языке имени немецкого поэта Генриха Гейне (Heinrich
Heine) использовалась транслитерация.

\section{Графематический анализ; токенизация; основные проблемы ГА}
\textsl{Графематический анализ} текста состоит в~следующем:
\begin{enumerate}
    \item разделение входного текста на~элементы (слова,
    разделители и~т.п.);
    \item удаление нетекстовых элементов;
    \item выделение и оформление нестандартных элементов, например:
    \begin{itemize}
        \item элементов форматирования~--- жирность, курсивность,
        подчеркивание;
        \item структурных элементов текста~--- заголовков, абзацев,
        примечаний;
        \item различных элементов текста, не~являющихся словами
        (числа, даты в~цифровых форматах, буквенно-цифровые комплексы
        и~т.п.);
        \item имен~(ИО), написанных инициалами;
        \item иностранных лексем, записанных латиницей и~т.п.
    \end{itemize}
\end{enumerate}

Cложности:
\begin{itemize}
    \item обработка дефиса и пробела;
    \item выделение составных предлогов, устойчивых оборотов, аналитических форм и~др.;
    \item иноязычные фрагменты;
    \item нетекстовые элементы.
\end{itemize}

\section{Форматы представления данных в лингвистике; форматы структуры (позиционные, с ключевыми словами, со справочником ISO 2709)}
\subsection{Форматы представления данных в лингвистике.}
\begin{itemize}
    \item Корпуса
    \begin{itemize}
        \item XML
        \item TEI
    \end{itemize}
    \item Онтологии
    \begin{itemize}
        \item RDF
        \item OWL
        \item \dots
    \end{itemize}
\end{itemize}

\subsection{Форматы структуры}
% нужны примеры!!!
\begin{itemize}
    \item позиционные форматы
    \item форматы с~ключевыми словами
    \item форматы со~справочником
\end{itemize}

\subsection{ISO 2709}
\label{subsec:ISO}
Стандарт \textbf{ISO 2709}~--- это универсальный стандарт
библиографического описания, предназначенный для~передачи
библиографической информации. Запись в~этом формате предполагает
наличие трех полей:
\begin{enumerate}
    \item маркер записи;
    \item справочник~--- содержит метки для~каждого поля
    (индекс поля, начальная позиция данных, длина содержимого поля);
    \item поля данных переменной длины;
\end{enumerate}

\textbf{Поля, обязательные для всех записей}
\\
001 ИДЕНТИФИКАТОР ЗАПИСИ \\
100 ДАННЫЕ ОБЩЕЙ ОБРАБОТКИ \\
200 ЗАГЛАВИЕ И СВЕДЕНИЯ ОБ ОТВЕТСТВЕННОСТИ (подполе \$a обязательно для каждой записи) \\
801 ИСТОЧНИК ЗАПИСИ \\

\section{Языки разметки (SGML, HTML, XML), их особенности}
x
\section{Форматы наполнения; UNIMARC}
Формат MARC (Machine Readable Cataloging) был разработан Библиотекой Конгресса США в 1965\nobreakdash-1966 годах с целью получения данных каталогизации в машиночитаемой форме. Аналогичная работа выполнялась в Великобритании Советом по Британской национальной каталогизации для обеспечения использования машиночитаемых данных при подготовке печатного издания Британской национальной библиографии~--- British national Bibliography (проект BNB MARC). На основе указанных разработок в 1968 г. начал создаваться коммуникативный англо\nobreakdash-американский формат MARC (проект MARC II). Целями его создания стало обеспечение: гибкости решения каталогизационных и других библиотечных задач и пригодности для национального библиографического описания любых видов документов и использования структуры записи в автоматизированных системах. Позднее USMARC превратился в комплекс специализированных форматов (USMARC Concise Formats) для записи библиографических, авторитетных, классификационных данных, данных о фондах и общественной информации, которые были с трудом совместимы.

Для преодоления несовместимости был разработан формат UNIMARC (Universal Machine Readable Cataloging). Предполагалось, что этот формат должен стать посредником между любыми национальными версиями форматов MARC и, следовательно, обеспечивать конвертирование данных из национального формата в него, а из него~--- в другой национальный формат.

Формат UNIMARC (MARC21) следует стандарту ISO~2709 (см. \hyperref[subsec:ISO]{соответствующий вопрос}).

\section{Форматы наполнения; рекомендации TEI}
\label{sec:TEI}
Кроме форматов структуры, которые описывают правила представления структуры, существуют стандарты, регламентирующие заполнение этой структуры,~--- форматы наполнения.
Система кодирования текстов (TEI) направлена на обеспечение обмена информацией, хранимой
в электронной форме. Основное внимание уделяется текстовой информации, но предусмотрены средства и для других форм, например, для графических изображений и звуковой информации (всего около двадцати). Все тексты в формате TEI содержат заголовок TEI (размечаемый как элемент \texttt{<teiHeader>}) и собственно текст (размечаемый как элемент \texttt{<text>}). Заголовок \texttt{teiHeader} должен содержать следующие элементы:
\begin{itemize}
\item \texttt{<fileDesc>} --- полное библиографическое описание файла;
\item \texttt{<encodingDesc>} --- описывает отношение между текстом и его источником;
\item \texttt{<profileDesc>} --- металингвистическое описание текста (язык, ситуация, в которой текст был создан и т.д.);
\item \texttt{<revisionDesc>} --- история правок файла.
\end{itemize}
Элементы, которые размечаются в документах всех типов:
\begin{itemize}
\item абзацы
\item знаки препинания
\item выделения и цитирование
\item редакторские правки
\item имена, адреса и другие именованные сущности
\item рисунки и прочие нетекстовые элементы
\item списки
\item примечания
\item ссылки
\end{itemize}
Текст TEI может быть монолитным (отдельное произведение) или объединенным (набор
отдельных произведений, \texttt{teiCorpus}). В любом случае текст может иметь
необязательные вводную часть и закрывающую часть. Между ними располагается основная часть
текста, которая, в случае объединенного текста, может состоять из групп, а они, в свою очередь, из других групп или текстов. Простой документ TEI на текстовом уровне состоит из следующих
элементов:
\begin{itemize}
\item \texttt{<front>} содержит различную вступительную информацию (заголовки, титульный лист,
предисловия, посвящения и т.п.), которую размещают перед основным текстом;
\item \texttt{<group>} содержит набор монолитных текстов или групп текстов;
\item \texttt{<body>} содержит всю основную часть одного монолитного текста, исключая то, что
относится к вводной или закрывающей частям текста;
\item \texttt{<back>} содержит различные приложения и т.п., которые располагаются после основной
части текста.
\end{itemize}
Как отдельные специфические виды текстов размечаются стихотворные и драматические произведения, а также транскрипции звучащей речи.

\section{Разметка корпусов, в том числе в TEI}

См. \hyperref[sec:TEI]{предыдущий вопрос}?

\section{Семантический уровень анализа языка; семантические проблемы 
в~традиционной лингвистике}
\textsl{Семантика}~--- раздел лингвистики (в частности, семиотики), 
изучающий смысловое значение единиц языка. В~качестве инструмента 
изучения применяют семантический анализ.

Непосредственно наблюдаемая ячейка семантики~--- полнозначное слово 
(например, существительное, глагол, наречие, прилагательное)~--- 
организована по~принципу «семантического треугольника»: внешний 
элемент~--- последовательность звуков или графических знаков 
(означающее)~--- связан в~сознании и в~системе языка, с~одной стороны, 
с~предметом действительности (вещью, явлением, процессом, признаком), 
называемым в~теории семантики {\sl денотатом}, референтом, с~другой стороны~--- 
с~понятием или представлением об~этом предмете, называемым смыслом, 
{\sl сигнификатом}, интенсионалом, означаемым.

Другой универсальной ячейкой семантики является предложение (высказывание), 
в~котором также выделяются денотат (или референт) как обозначение факта 
действительности и сигнификат (или смысл), соответствующий суждению об~этом 
факте. Денотат и сигнификат в~этом смысле относятся к~предложению в~целом. 
В~отношении~же частей предложения обычно подлежащее (или субъект) денотатно, 
референтно, а сказуемое (или предикат) сигнификатно.

Аналогично слову и предложению организована семантика всех единиц языка. 
Она распадается на~две сферы~--- предметную, или денотатную (экстенсиональную), 
семантику и сферу понятий, или смыслов,~--- сигнификатную (интенсиональную) 
семантику. Термины <<экстенсиональная семантика>> и <<интенсиональная семантика>> 
восходят к~описанию отдельного слова-понятия, где ещё в традиции средневековой 
логики объём понятия (т.е. объём его приложений к~предметам, покрываемая 
предметная область) назывался термином extensio ‘растяжение’, а содержание 
понятия (т.е. совокупность мыслимых при~этом признаков)~--- словом intensio 
‘внутреннее натяжение’. Денотатная и сигнификатная сферы семантики в~естественных 
языках (в отличие от некоторых специальных искусственных языков) строятся 
довольно симметрично, при~этом сигнификатная (понятийная) в~значительной 
степени копирует в~своей структуре денотатную (предметную) сферу. Однако 
полный параллелизм между ними отсутствует, и ряд ключевых проблем семантики 
получает решение только применительно к~каждой сфере в~отдельности. Так, 
предметная, или денотатная, синонимия, экстенсиональное тождество языковых 
выражений не~обязательно влекут за~собой сигнификатную, или понятийную, 
синонимию, интенсиональное тождество, и наоборот. 

Семантические отношения описываются семантикой как разделом языкознания 
с~разных точек зрения. К~{\sl парадигматике} относятся группировки слов в~системе 
языка, основой которых выступает оппозиция,~--- синонимия, антонимия, гипонимия, 
паронимия, гнездо слов, семья слов, лексико-семантическая группа, а также 
наиболее общая группировка слов~--- поле. Различаются поля двух основных 
видов: 
\begin{enumerate}
    \item объединения слов по их отношению к~одной предметной области~--- 
    предметные, или денотатные, поля, например цвето-обозначения, имена 
    растений, животных, мер и весов, времени и~т.д.; 
    \item объединения слов по~их отношению к~одной сфере представлений или 
    понятий~--- понятийные, или сигнификатные, поля, например обозначения 
    состояний духа (чувств радости, горя, долга), процессов мышления, 
    восприятия (видения, обоняния, слуха, осязания), возможности, необходимости 
    и~т.п. 
\end{enumerate}
В~предметных полях слова организованы преимущественно по~принципу <<пространство>>
и по~принципам соотношения вещей: часть и целое, функция (назначение) и ее 
аргументы (производитель, агенс, инструмент, результат); в~понятийных полях~--- 
преимущественно по принципу <<время>> и по~принципам соотношения понятий 
(подчинение, гипонимия, антонимия и др.). Парадигматические отношения 
формализуются с~помощью математической теории множеств.

К~{\sl синтагматике} относят группировки слов по~их расположению в~речи 
относительно друг друга (сочетаемость, аранжировка). Основой этих отношений 
выступает {\sl дистрибуция} (см.~Дистрибутивный анализ). Они формализуются 
с~помощью математической теории вероятностей, статистико-вероятностного 
подхода, исчисления предикатов и исчисления высказываний, теории алгоритмов.

При~соотнесении результатов описания семантики в~парадигматике и синтагматике 
выявляются некоторые их общие черты, наличие семантических инвариантов, 
а также более мелкие и более универсальные, чем слово, семантические 
единицы~--- семантические признаки, или семы (называемые также компонентом, 
иногда семантическим параметром или функцией). Основные семы в~лексике совпадают 
с~категориальными грамматическими значениями в~грамматике (граммемы). 
В~парадигматике сема выявляется как минимальный признак оппозиции, 
а в~синтагматике~--- как минимальный признак сочетаемости.

Одна из~важнейших проблем семантики~--- системность лексических значений~--- 
имеет доступ к~своему исследованию со~стороны синтаксиса, что позволяет 
объективным путем, посредством использования внутриязыковых критериев, 
обнаруживать те семантические связи между словами, с~помощью которых данные 
слова образуют лексические подсистемы, поля, группы, т.е. совокупности 
слов, имеющие семантическую общность. Так, например, лишь существительные 
типа вид, семейство, класс, разряд, категория, группа, разновидность, род 
и~т.п. допускают следующую трансформацию: этот вид (семейство, класс, разряд 
и.~т.п.)~--- объекты этого вида (семейства, класса, разряда и т.п.); только 
существительные со~значением вместилища допускают трансформацию типа {\it банка 
из-под~варенья}~--- {\it банка под~варенье}, {\it бутылка из-под~молока}~--- 
{\it бутылка под~молоко} и~т.п.; только существительные со~значением параметра вещей 
допускают трансформацию типа {\it высотой с~дом}~--- {\it высокий, как дом}, 
{\it шириной с~улицу}~--- {\it широкий, как улица} и~т.п. Поскольку наиболее 
активными с~синтаксической точки зрения являются глаголы, исследование их 
семантики служит удачной сферой приложения дистрибутивно-трансформационного 
метода с~целью, например, таксономии глаголов с~семантической точки 
зрения, что и осуществлено на~материале русского языка.

\section{Компьютерная семантика. Ее отличия от традиционной лингвистической.}
Семантический (смысловой) анализ текста~--- одна из~ключевых проблем 
как теории создания систем искусственного интеллекта, относящаяся 
к~обработке естественного языка, так и компьютерной лингвистики. 
Результаты семантического анализа могут применяться для~решения задач 
в~таких областях как, например, психиатрия (для диагностирования 
больных), политология (предсказание результатов выборов), торговля 
(анализ <<востребованности>> тех или иных товаров на~основе комментариев 
к~данному товару), филология (анализ авторских текстов), поисковые 
системы, системы автоматического перевода и~т.д.

Несмотря на~свою востребованность практически во~всех областях жизни 
человека, семантический анализ является одной из~сложнейших математических 
задач. Вся сложность заключается в~том, чтобы <<научить>> компьютер 
правильно трактовать образы, которые автор текста пытается передать 
своим читателям/слушателям.

Способность <<распознавать>> образы считается основным свойством 
человеческих существ, как, впрочем, и других живых организмов. 
Образ представляет собой описание объекта. В~каждое мгновение 
нашего бодрствования мы совершаем акты распознавания. Мы опознаем 
окружающие нас объекты и в~соответствии с~этим перемещаемся и 
совершаем определенные действия.

Естественный язык в~отличие, например, от~компьютерных (алгоритмических) 
языков формировался во~многом стихийно, не~формализовано. Это 
обуславливает целый ряд сложностей в~понимании текста, вызванных, 
например, неоднозначным толкованием одних и тех~же слов в~зависимости 
от~контекста, который может быть и неизвлекаем, в~принципе, из~самого 
текста. Следовательно, этот контекст или знание о~предметной области 
в~систему должны быть заранее внесены. К~тому~же зачастую практические 
задачи требуют точного определения времени, места того, что описано 
в~тексте, точной идентификации людей и~т.д., в~то время как подобная 
информация находится за~пределами данного текста. В~этом случае система 
может или не~обрабатывать эту информацию, или оставить ее до~выяснения 
контекста и даже попытаться проявить инициативу в~выяснении контекста, 
например, в~диалоге с~оператором, задающим ввод текста. То, как ведет 
себя система в~подобной ситуации, определяется стилем и схемой работы 
системы.

Промышленные системы автоматической обработки текста, в~основном, сейчас 
используют два этапа анализа текста: морфологический и синтаксический. 
Однако теоретические разработки многих исследователей предполагают 
существование следующего за~синтаксическим этапа~--- семантического. 
В~отличие от~предыдущих шагов семантический этап использует формальное 
представление смысла составляющих входной текст слов и конструкций. 
Суть семантического анализа понимается разными исследователями по-разному. 
Многие ученые сходятся во~мнении, что в~сферу семантического анализа входит:
\begin{itemize}
    \item построение семантической интерпретации слов и конструкций;
    \item установление <<содержательных>> семантических отношений 
    между элементами текста, которые уже принципиально не~ограничены 
    размером одного слова.
\end{itemize}

Основные проблемы понимания текста в~обработке естественных языков таковы:
\begin{enumerate}
    \item Знание системой контекста и проблемной области и обучение 
    этому системы. Например, из~предложения <<мужчина вошел в~дом 
    с~красным портфелем>> можно извлечь как представление о~мужчине 
    с~красным портфелем, так и о~доме с~красным портфелем, если 
    заранее не~иметь в~виду, что применительно к~мужчинам употребление 
    принадлежности портфеля гораздо вероятнее, чем применительно к~дому.
    \item Различная форма передачи синтаксиса (т.е. структуры) предложения 
    в~разных языках. Например, если синтаксическая роль слова (подлежащее, 
    сказуемое, определение и~т.д.) в~английской речи во~многом определяется 
    положением слова в~предложении относительно других слов, то в~русском 
    предложении существует свободный порядок слов и для~выявления 
    синтаксической роли слова служат его морфологические признаки 
    (например, окончания слов), служебные слова и знаки препинания.
    \item Проблема равнозначности. Предложения <<длинноухий грызун 
    бросился от~меня наутек>> и <<заяц бросился от~меня наутек>> 
    могут означать одно и то~же, но могут иметь и разный смысл, 
    например, если в~первом случае имелся в~виду длинноухий 
    тушканчик.
    \item Наличие в~тексте новых для~компьютера слов, например неологизмов. 
    Самообучаемая система должна уметь <<интуитивно>> определить 
    (возможно, и неправильно, но с~возможность в~дальнейшем исправить себя) 
    лексическую роль, морфологическую форму этого слова, попробовать вписать 
    его в~существующую структуру знаний, наделить его какими-то атрибутами 
    или выяснить все это в~диалоге с~оператором. Система, не~способная 
    к~самообучению просто потеряет какое-то количество информации.
    \item Проблема совместимости новой информации с~уже накопленными 
    знаниями. Новая информация может каким-то образом противоречить 
    уже накопленной информации. Необходимо реализовать механизм, определяющий, 
    в~каких случаях нужно отвергнуть старую информацию, а в~каких~--- новую.
    \item Проблема временных противоречий. Так в~предложении <<я думал, 
    что сверну горы>> глагол в~прошедшей форме <<думал>> сочетается 
    с~глаголом будущего времени <<сверну>>.
    \item Проблема эллипсов, то есть предложений с~пропущенными фактически, 
    но существующими неявно благодаря контексту словами. Например, 
    в~предложении <<я передам пакет тебе, а ты~--- Ивану Петровичу>> 
    во~второй части опущен глагол <<передашь>> и существительное <<пакет>>.
\end{enumerate}

Лингвистический процессор может быть интегрирован с~системой распознавания 
и (или) синтеза речи, что может сделать процесс общения с~компьютером 
максимально удобным, а, следовательно, и продуктивным.

Одной из~наиболее очевидных направлений применения лингвистических 
процессоров является машинный перевод с~одного естественного языка (ЕЯ) 
на~другой.

Системы семантического анализа не~могут существовать без~морфологической 
составляющей. В~качестве морфологической составляющей выступают различные 
виды словарей словоформ (т.е. содержащие все варианты склонения, спряжения 
и~т.д. того или иного слова). Но возникает проблема <<неполноты>> того или 
иного словаря. Существует ряд подходов для~решения этой проблемы.

Системы семантического анализа не~могут существовать и без~синтаксической 
составляющей. Основной задачей синтаксического анализа является построение 
синтаксического дерева предложения. Также как и морфологический анализ, 
синтаксический анализ является предварительным этапом перед семантическим 
анализом. На этом этапе отсеивается большая часть омонимов (слова разного 
значения, но одинаково звучащие, напр., пол, коса, ключ), выявленных на~этапе 
морфологического анализа. Что, в~свою очередь, существенно ускорит 
семантический анализ.

Для~представления в~памяти компьютера значения всех содержательных единиц 
рассматриваемого языка (лексических, морфологических, синтаксических и 
словообразовательных) и приведения их к~единому, формальному виду, 
понятному компьютеру, используется, специально созданный для~этого 
искусственный язык или, как его еще называют некоторые ученые, метаязык.

\section{Семантические категории. Понятие.}
{\sl Семантическая категория}~--- это класс выражений с~однотипными 
предметными значениями, при~этом включающий все выражения с~предметным 
значением данного типа.

Такими классами являются имена, предикаторы, предметные функторы, 
логические термины, повествовательные предложения.

Имена~--- слова и словосочетания, являющиеся знаками предметов.

Предикаторы~--- выражения языка (слова и словосочетания), предметными 
значениями которых являются свойства (одноместные предикаторы) 
и отношения (многоместные предикаторы).

Свойствами в~современной логике называют характеристики отдельных предметов 
(<<белый>>, <<странный>>, <<иметь спинку>>, <<ходить>> и~т.п.). 
Отношения~--- это связь между двумя и более предметами (<<находиться 
между>>, <<быть братом>>, <<быть больше>>, <<знать лучше, чем>> и~т.п.). 
Таким образом, отношения представляют собой характеристики не~отдельных 
предметов, а некоторых систем предметов.

Наличие или отсутствие у какого-либо предмета свойства или отношения 
к~другим предметам называется признаком. Признаки~--- это любые возможные 
характеристики предмета, все, что можно высказать о~предмете.

Предметные функторы~--- это знаки так называемых предметных функций. 
Наряду с~математическими функциями сюда относятся такие особые 
характеристики предметов, как скорость, плотность, возраст, пол, 
профессия, агрегатное состояние, место жительства и др. 
Иногда их называют предметно-функциональными характеристиками.

Логические термины (логические константы)~--- это знаки логических 
отношений <<и>>, <<или>>, <<если\dots, то\dots>>, <<неверно, что>> 
и операций <<всякий>>, <<существует>> (<<некоторые>>), <<тот…, который…>>.

\section{Синтагматические отношения между понятиями.}
x
\section{Парадигматические отношения между понятиями.}
x
\section{Понятие и слово.}
x
\section{Краткий обзор языков представления знаний.}
% https://ru.wikipedia.org/wiki/Представление_знаний

\section{Предикаты}
x
\section{Статья Мочаловой.}
x
\section{Статья Шерстиновой.}
\textbf{Речевой корпус <<Один речевой день>> (ORD корпус)} разрабатывается с целью исследования повседневной устной речи и бытовой коммуникации. Методологической основой создаваемого корпуса является осуществление звукозаписей повседневной речи в условиях, максимально приближенных к естественным, для чего используется методика непрерывной 24\nobreakdash-часовой записи всей речевой коммуникации информантов в течение суток. К настоящему времени записано более 300 часов звучания, полученных от 40 информантов (20 мужчин и 20 женщин). Звукозаписи переформатированы, убраны длительные (больше 5 минут) шумовые фрагменты, не содержащие речи. Звукозаписи разрезаны на коммуникативные эпизоды по принципу общих условий коммуникации и качества звукозаписи. В результате было получено более 900 файлов-эпизодов.

Для аннотирования корпуса ORD используются два профессиональных программных продукта:
\begin{itemize}
\item программа многоуровневого лингвистического аннотирования \textbf{ELAN},
\item программа профессионального фонетического анализа \textbf{Praat}.
\end{itemize}

Первичное аннотирование (расшифровка) данных осуществляется в программе ELAN и предполагает заполнение следующих уровней:
\begin{itemize}
\item Frase --- отделение реплик говорящих от неречевого сигнала,
\item Speaker --- кодирование говорящего,
\item Voice --- определение качества голоса,
\item Events --- разметка неречевых аудиособытий,
\item FonetComment --- отклонения от литературной нормы,
\item FraseComment --- информация о реализации конкретной реплики,
\item Notes --- общий комментарий,
\item Episode --- обозначение мелких эпизодов и мини\nobreakdash-сценариев.
\end{itemize}


\section{Статья Мазова.}
x
\section{Статья Маркова.}
x

\end{document}
