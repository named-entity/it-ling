% ----------------------------------------------------------------
% Article Class (This is a LaTeX2e document)  ********************
% ----------------------------------------------------------------
\documentclass[12pt]{article}
\usepackage[english, russian]{babel}
\usepackage{amsmath,amsthm}
\usepackage{amsfonts}
\usepackage{indentfirst}
\usepackage{enumitem}
\usepackage[colorlinks,urlcolor=blue]{hyperref}
% THEOREMS -------------------------------------------------------
\newtheorem{thm}{Theorem}[section]
\newtheorem{cor}[thm]{Corollary}
\newtheorem{lem}[thm]{Lemma}
\newtheorem{prop}[thm]{Proposition}
\theoremstyle{definition}
\newtheorem*{defn}{Определение}
\theoremstyle{remark}
\newtheorem{rem}[thm]{Remark}
\numberwithin{equation}{section}
% ----------------------------------------------------------------
\begin{document}

\title{Информационные технологии в лингвистике}%
\author{Билеты к экзамену}%
\date{}
% ----------------------------------------------------------------
\maketitle
% ----------------------------------------------------------------
\section{Лингвистические информационные технологии; лингвистические
ресурсы в Интернете}
\subsection{Информационные технологии в лингвистике}
В~70-е годы XX века началось бурное развитие компьютерных технологий.
Компьютеры позволили обрабатывать огромные массивы информации, прежде
неподвластные обработке. Это дало скачок в~развитии некоторых областей
прикладной лингвистики, ранее находившихся в~зачаточном состоянии, и
позволило по-новому взглянуть на~ряд фундаментальных задач.
\begin{defn}
    {\sl Информационные технологии в лингвистике}~--- это совокупность
    законов, методов и средств получения, хранения, передачи,
    распространения, преобразования информации о~языке и законов 
    его функционирования с~помощью компьютеров.
\end{defn}

Информационные технологии находят свое применение прежде всего
в~прикладных задачах. К~их числу можно отнести:
\begin{enumerate}
    \item создание систем искусственного интеллекта;
    \item создание систем автоматического перевода;
    \item создание систем автоматического аннотирования и реферирования
    текстов;
    \item создание систем порождения текстов;
    \item создание систем обучения языку;
    \item создание систем понимания устной речи;
    \item создание систем генерации речи;
    \item создание автоматизированных информационно-поисковых систем;
    \item создание систем атрибуциии дешифровки анонимных и
    псевдоанонимных текстов;
    \item разработка различных баз данных (словарей, карточек, каталогов
    реестров, и~т.п.) для гуманитарных наук;
    \item разработка различного типа автоматических словарей;
    \item разработка систем передачи информации в~сети Интернет и~т.п.
\end{enumerate}

Самое слово {\sl информация} означает некоторые сведения о~внешнем и
внутреннем мире, которые мы используем для~регулирования своего 
поведения. Она определяется такими параметрами: ценность, достоверность, 
полнота, актуальность, логичность, компактность.

С~современной точки зрения {\sl информатика}~--- наука о~законах и 
методах получения, хранения, передачи, распространения и использования 
информации в~естественных и искусственных системах с~применением 
компьютера.

\subsection{Лингвистические ресурсы в интернете}
Вот некоторые лингвистические ресурсы в интернете:
\begin{enumerate}[label=\*]
    \item \href{http://linguistlist.org/}{The Linguist List}~--- большой
    каталог программного обеспечения для~различных областей компьютерной
    обработки текстов и лингвистики.
    \item \url{http://www.sil.org/computing/}~--- обширный каталог 
    программ по~вычислительной лингвистике, разработанных в~рамках 
    проекта SIL.
    \item \url{http://www.sil.org/linguistics/computing.html}~--- очень 
    большая коллекция ссылок на~программы чрезвычайно широкой лингвистической
    направленности в~сети Internet.
    \item \url{http://www.content-analysis.de/index.html}~--- электронный
    каталог ресурсов, связанных с~анализом текстов (на~англ.языке).
    \item \href{http://www.gramota.ru/index.html}{Справочно-информационный
    портал <<Русский язык>>}~--- замечательный ресурс для~истинных 
    любителей русской словесности, содержащий массу полезной информации. 
    Также включает on-line словари русского языка.
    \item \href{http://www.textanalysis.info/}{Text Analysis Info}~--- 
    бесплатный информационный портал, посвященный анализу контента 
    межчеловеческого общения. Также представлены различные программы, 
    предназначенные для обработки таких источников как аудио-, видео- или 
    речевых данных.
    \item \href{http://www.lti.cs.cmu.edu/research/projects}{LTI Projects}~---
    каталог проектов, посвященных созданию систем машинного перевода, 
    обработки речи, информационного поиска, извлечения знаний и других.
    \item \url{http://ruscorpora.ru/}~--- национальный корпус русского языка.
    \item \url{http://opencorpora.org/}~--- открытый корпус русского языка.
\end{enumerate}

\section{Типология программных средств для лингвистических информационных 
технологий}
Существует немало программных средств, предназначенных для~обеспечения 
анализа, обработки, хранения и поиска текстов, рисунков и аудиоданных 
на~естественном языке. 
\paragraph{Программы анализа и лингвистической обработки текстов.}
Наиболее важные парсеры этого типа производят морфологическую разметку
текста, по~возможности снимая неоднозначность на~разных уровнях, или
строят графы зависимостей, выделяют имена собственные и решают более
узконаправленные задачи. 

\paragraph{Программы преобразования текстов.}
Решают задачи поиска и замены в~тексте определенных элементов,
смены кодировки, обработку разметки и т.п.
\paragraph{Проверка орфографии.}
Ряд парсеров проверяет пунктуацию, орфографию, грамматику и оценивает
стилистику текста.
\paragraph{Автоматические генераторы текстов}
Еще некоторое время назад многие подобные программы имели шуточный
характер: они генерировали стихи или <<философские>> изречения,
но сейчас и это направление бурно развивается. Крупные компании
используют автоматические генераторы канцелярских документов, отчетов
и даже автореферирование и рецензирование. 
\paragraph{Машинный перевод.}
Программы и сервисы, предназначенные для~автоматического перевода
текстов с~одного естественного языка на~другой. 
\paragraph{Словари и тезаурусы}
Электронные словари, теперь бывают и статистические.
\paragraph{Поисковые машины и системы полнотекстового поиска.}
Поиск по~ключевой информации, кластеризация результатов.
\paragraph{Синтез и распознавание речи.}
Направление исследований: распознавание $\rightarrow$ перевод 
$\rightarrow$ синтез. 
\paragraph{Распознавание символов}

\section{Автоматический анализ текста; уровни анализа; взаимодействие 
между уровнями}
Отражение уровневой организации языка в~архитектуре систем 
компьютерного анализа естественно-языкового текста:
\begin{enumerate}[label=---]
    \item фонетический уровень;
    \item графематический уровень;
    \item морфологический уровень;
    \item синтаксический уровень;
    \item семантический уровень.
\end{enumerate}

\paragraph{Фонетика}~--- раздел языкознания, изучающий речевые 
звуки и звуковое строение языка (слоги, звукосочетания, закономерности 
соединения звуков в речевую цепочку). Различает звуки гласные 
и согласные, звонкие и глухие и т.д. Изучает:
\begin{enumerate}[label=---]
    \item звук речи с~точки зрения его создания, какие органы речи 
    участвуют в~его произношении;
    \item звук как колебание воздуха и фиксирует его физические 
    характеристики: частоту (высоту), силу (амплитуду), длительность;
    \item функции звуков в~языке, оперирует фонемами, т.е. 
    минимальными языковыми единицами, обладающими смыслоразличительной 
    функцией.
\end{enumerate}
\paragraph{Графематический анализ.} Этап графематического 
анализа предназначен для~выделения элементов структуры текста: 
параграфов, абзацев, предложений, отдельных слов и~т.д. 
В~задачу графематического анализа входят:
\begin{enumerate}[label=---]
    \item Выделение абзацев, заголовков, примечаний;
    \item Выделение предложений из~входного текста;
    \item Разделение входного текста на~слова, цифровые комплексы, 
    формулы и~т.д. {\sl Токенизация};
    \item Сборка слов, написанных в~разрядку;
    \item Выделение устойчивых оборотов, не~имеющих словоизменительных 
    вариантов;
    \item Выделение ФИО (фамилия, имя, отчество), когда имя и отчество 
    написаны инициалами;
    \item Выделение иностранных лексем, записанных латиницей;
    \item Выделение электронных адресов и имен файлов;
\end{enumerate}

\paragraph{Морфологический анализ.} Единица~--- слово. На~этом этапе 
обрабатываются отдельные слова, в~них выделяются основы и флексии 
(изменяемые части слов)~--- приставки, суффиксы, окончания. 
В~дальнейшем флексии используются для~установления грамматических 
отношений между словами в~рамках одного предложения.

Слова как единицы грамматические и лексические группируются 
в~части речи, т.е. в грамматические классы слов, объединяющиеся 
на~основании обобщенных значений. Обобщенное значение, характеризующее 
все слова той или иной части речи, есть абстрактное представление 
того общего, что присутствует в~лексических и морфологических значениях 
конкретных слов данного класса. Наиболее обобщенными значениями 
для~частей речи являются значения предмета (субстанции) и признака~--- 
процессуального (представляемого как действие или состояние) и 
непроцессуального (представляемого как качество или свойство).

Морфологический анализ обеспечивает определение нормальной
формы, от~которой была образована данная словоформа, и набора 
параметров, приписанных данной словоформе. Нормальная форма 
(именительный падеж для~существительных, инфинитив для~глаголов 
и~т.д.) называется {\sl леммой}, а сам процесс определения лемм~--- 
{\sl лемматизацией}.

\paragraph{Синтаксический анализ.} Единица~--- предложение.
В~результате синтаксического анализа линейная последовательность
токенов (слов) преобразуется в~набор синтаксических отношений. 
Грамматично построенные предложения являются связными, т.е. 
лишенными разрывов в~цепочке синтаксических отношений. Отношения 
являются бинарными. Синтаксическое отношение неравноправно: 
определяемое слово <<главнее>> своего определения. Важная 
особенность синтаксических зависимостей заключается в~том, 
что они далеко не~всегда связывают слова, находящиеся рядом 
в~цепочке.

Выполнение задачи осложняется огромным количеством альтернативных 
вариантов, возникающих в~ходе разбора, связанных как 
с~многозначностью входных данных (одна и та же словоформа может 
быть получена от~различных нормальных форм), так и неоднозначностью 
самих правил разбора.

\paragraph{Семантический анализ.} Занимается решением задач, 
связанных с~возможностью определения значения слова в~зависимости 
от~контекста и конкретной ситуации, понимания смысла фразы. 
Элемент значения языкового знака называется {\sl семой}.

Во многих случаях смысловой элемент состоит из нескольких слов.
Последовательность из двух или более слов, частотность совместного 
появления которых в тексте выше, чем ожидаемая вероятность их 
совместного появления, называется {\sl коллокацией}. В отличие 
от свободного словосочетания, коллокация определяет, какие слова 
могут быть использованы вместе, например, с какими предлогами 
управляет тот, или иной глагол, или какие глаголы и существительные 
обычно используются вместе.

С помощью компьютерных технологий коллокации могут автоматически 
извлекаться из текстов. Для этого используются различные меры 
ассоциативной связи, которые оценивают, является ли взаимное появление
лексических единиц случайным, или оно статистически значимо. Однако
часто статистически значимое совместное появление двух слов не образует
коллокации.

\end{document} 
